\documentclass[12pt, titlepage]{article}

\usepackage{amsmath, mathtools}

\usepackage[round]{natbib}
\usepackage{amsfonts}
\usepackage{amssymb}
\usepackage{graphicx}
\usepackage{colortbl}
\usepackage{xr}
\usepackage{hyperref}
\usepackage{longtable}
\usepackage{xfrac}
\usepackage{tabularx}
\usepackage{float}
\usepackage{siunitx}
\usepackage{booktabs}
\usepackage{multirow}
\usepackage[section]{placeins}
\usepackage{caption}
\usepackage{fullpage}

\hypersetup{
bookmarks=true,     % show bookmarks bar?
colorlinks=true,       % false: boxed links; true: colored links
linkcolor=red,          % color of internal links (change box color with linkbordercolor)
citecolor=blue,      % color of links to bibliography
filecolor=magenta,  % color of file links
urlcolor=cyan          % color of external links
}

\usepackage{array}

\externaldocument{../../SRS/SRS}

%% Comments

\usepackage{color}

%\newif\ifcomments\commentstrue %displays comments
\newif\ifcomments\commentsfalse %so that comments do not display

\ifcomments
\newcommand{\authornote}[3]{\textcolor{#1}{[#3 ---#2]}}
\newcommand{\todo}[1]{\textcolor{red}{[TODO: #1]}}
\else
\newcommand{\authornote}[3]{}
\newcommand{\todo}[1]{}
\fi

\newcommand{\wss}[1]{\authornote{blue}{SS}{#1}} 
\newcommand{\plt}[1]{\authornote{magenta}{TPLT}{#1}} %For explanation of the template
\newcommand{\an}[1]{\authornote{cyan}{Author}{#1}}

%% Common Parts

\newcommand{\progname}{Mechatronics Engineering} % PUT YOUR PROGRAM NAME HERE
\newcommand{\authname}{Team 25, Preliminary
\\ Ahmed Nazir, nazira1
\\ Stephen Oh, ohs9
\\ Muhanad Sada, sadam
\\ Tioluwalayomi Babayeju, babayejt} % AUTHOR NAMES                  

\usepackage{hyperref}
    \hypersetup{colorlinks=true, linkcolor=blue, citecolor=blue, filecolor=blue,
                urlcolor=blue, unicode=false}
    \urlstyle{same}
                                


\begin{document}

\title{Module Interface Specification for \progname{}}

\author{\authname}

\date{\today}

\maketitle

\pagenumbering{roman}

\section{Revision History}

\begin{tabularx}{\textwidth}{p{3cm}p{2cm}X}
\toprule {\bf Date} & {\bf Version} & {\bf Notes}\\
\midrule
Date 1 & 1.0 & Notes\\
Date 2 & 1.1 & Notes\\
\bottomrule
\end{tabularx}

~\newpage

\section{Symbols, Abbreviations and Acronyms}

See SRS Documentation at \wss{give url}

\wss{Also add any additional symbols, abbreviations or acronyms}

\newpage

\tableofcontents

\newpage

\pagenumbering{arabic}

\section{Introduction}

The following document details the Module Interface Specifications for
\wss{Fill in your project name and description}

Complementary documents include the System Requirement Specifications
and Module Guide.  The full documentation and implementation can be
found at \url{https://github.com/ahmed-nazir/Capstone}.  \wss{provide the url for your repo}

\section{Notation}

\wss{You should describe your notation.  You can use what is below as
  a starting point.}

The structure of the MIS for modules comes from \citet{HoffmanAndStrooper1995},
with the addition that template modules have been adapted from
\cite{GhezziEtAl2003}.  The mathematical notation comes from Chapter 3 of
\citet{HoffmanAndStrooper1995}.  For instance, the symbol := is used for a
multiple assignment statement and conditional rules follow the form $(c_1
\Rightarrow r_1 | c_2 \Rightarrow r_2 | ... | c_n \Rightarrow r_n )$.

The following table summarizes the primitive data types used by \progname. 

\begin{center}
\renewcommand{\arraystretch}{1.2}
\noindent 
\begin{tabular}{l l p{7.5cm}} 
\toprule 
\textbf{Data Type} & \textbf{Notation} & \textbf{Description}\\ 
\midrule
character & char & a single symbol or digit\\
integer & $\mathbb{Z}$ & a number without a fractional component in (-$\infty$, $\infty$) \\
natural number & $\mathbb{N}$ & a number without a fractional component in [1, $\infty$) \\
real & $\mathbb{R}$ & any number in (-$\infty$, $\infty$)\\
\bottomrule
\end{tabular} 
\end{center}

\noindent
The specification of \progname \ uses some derived data types: sequences, strings, and
tuples. Sequences are lists filled with elements of the same data type. Strings
are sequences of characters. Tuples contain a list of values, potentially of
different types. In addition, \progname \ uses functions, which
are defined by the data types of their inputs and outputs. Local functions are
described by giving their type signature followed by their specification.

\section{Module Decomposition}

The following table is taken directly from the Module Guide document for this project.

\begin{table}[h!]
\centering
\begin{tabular}{p{0.3\textwidth} p{0.6\textwidth}}
\toprule
\textbf{Level 1} & \textbf{Level 2}\\
\midrule

{Hardware-Hiding} & ~ \\
\midrule

\multirow{7}{0.3\textwidth}{Behaviour-Hiding} & Input Parameters\\
& Output Format\\
& Output Verification\\
& Temperature ODEs\\
& Energy Equations\\ 
& Control Module\\
& Specification Parameters Module\\
\midrule

\multirow{3}{0.3\textwidth}{Software Decision} & {Sequence Data Structure}\\
& ODE Solver\\
& Plotting\\
\bottomrule

\end{tabular}
\caption{Module Hierarchy}
\label{TblMH}
\end{table}

\newpage
~\newpage

\section{MIS \wss{Module Name}} \label{Module} \wss{Use labels for
  cross-referencing}

\wss{You can reference SRS labels, such as R\ref{R_Inputs}.}

\wss{It is also possible to use \LaTeX for hypperlinks to external documents.}

  \subsection{Module - ui\_main.py}

  \subsubsection{Description}
  Python file generated by PyQt designer which sets up the application’s window and its design

  \subsubsection{Classes}
  \textbf{Class:} Ui\_MainWindow() - Contains all methods for setting up the application’s window and its static front end design \\

    \noindent \begin{tabular}{| p{0.6\textwidth} | p{0.2\textwidth}| p{0.2\textwidth}|}
      \hline
      \rowcolor[gray]{0.9}
      Methods & Parameters & Return\\
      \hline
      setupUi() - Takes a PyQt MainWindow object and sets up it’s layout according to the ui file created in designer &Self, MainWindow [QMainWindow] & None \\
      \hline
      retranslateUi() - Sets the static text of the GUI’s buttons and labels & Self, MainWindow [QMainWindow] & None \\
      \hline
    \end{tabular}

    \subsection{Module - ui\_functions.py}

    \subsubsection{Description}
    Imports all necessary libraries for backend functions, creates connection to database, and contains class for UI functions
    
    \subsubsection{Classes}
    \textbf{Class:} UIFunctions() - Contains the functions that are connected to buttons in the application’s UI \\
    
      \noindent \begin{tabular}{| p{0.6\textwidth} | p{0.2\textwidth}| p{0.2\textwidth}|}
        \hline
        \rowcolor[gray]{0.9}
        Methods & Parameters & Return\\
        \hline
        toggleMenu() - Handles the animation for toggling the side menu &Self, maxWidth [integer], enable [boolean] & None \\
        \hline
        login\_into\_app() - Checks if the enter username/password are valid and correct and signs user into their account & Self & None \\
        \hline
        continue\_signup() - Checks if all the sign up fields are valid and stores account/login details in database & Self & None \\
        \hline
      \end{tabular}

  \subsubsection{Functions}
    \noindent \begin{tabular}{| p{0.6\textwidth} | p{0.2\textwidth}| p{0.2\textwidth}|}
      \hline
      \rowcolor[gray]{0.9}
      Function & Parameters & Return\\
      \hline
      hash\_new\_password() - Generates a hashed password based on the user’s inputted password & password [string] & salt [string], hashed\_pass [string] \\
      \hline
      is\_correct\_password() - Checks if inputted password matches stored password in database & salt\_hex [string], stored\_hash [string], pass\_to\_check [string] & Boolean \\
      \hline
    \end{tabular}

  \subsubsection{Exception Handling}
  Input validation of the user information is the main form of exception handling. User fields for signing up are checked to ensure that they are not empty and that the password follows the rules of having 8 minimum characters and includes an alphabet, number, and an non-alphanumeric character. When logging in, inputted passwords are checked to ensure that they match the passwords stored in the database. Users will see error messages in the GUI according to what they inputted incorrectly.

  \subsection{Module - main.py}

  \subsubsection{Description}
  Imports backend functions and frontend setup of GUI. This is also used to start and run the desktop application

  \subsubsection{Classes}
  \textbf{Class:} MainWindow() - Initializes a PyQt main window that is defined in ui\_main.py and connects the buttons in the desktop application’s UI to backend functions defined in ui\_functions.py \\

    \noindent \begin{tabular}{| p{0.6\textwidth} | p{0.2\textwidth}| p{0.2\textwidth}|}
      \hline
      \rowcolor[gray]{0.9}
      Methods & Parameters & Return\\
      \hline
      \_\_init\_\_() - Initializes the application and connects UI buttons to backend functions &  Self & None \\
      \hline
      changeText() - Add text to menu buttons when toggling full side menu and vice-versa & Self & None \\
      \hline
    \end{tabular}

  \subsection{Module - resource\_rc.py}

  \subsubsection{Description}
  Python file generated by PyQt resource compiler and sets up all the PyQt resources (local images) to be displayed during runtime of application
  
  \subsubsection{Classes}

  \subsubsection{Functions}
  
    \noindent \begin{tabular}{| p{0.6\textwidth} | p{0.2\textwidth}| p{0.2\textwidth}|}
      \hline
      \rowcolor[gray]{0.9}
      Function & Parameters & Return\\
      \hline
      qInitResources() - Registers the raw byte data of each image to the Qt resource system &  None & None \\
      \hline
      qCleanupResources() - Unregisters the raw byte data of each image to the Qt resource system & None & None \\
      \hline
    \end{tabular}


\newpage

\bibliographystyle {plainnat}
\bibliography {../../../refs/References}

\newpage

\section{Appendix} \label{Appendix}

\wss{Extra information if required}

\end{document}