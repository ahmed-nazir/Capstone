\documentclass[12pt, titlepage]{article}

\usepackage{fullpage}
\usepackage[round]{natbib}
\usepackage{multirow}
\usepackage{booktabs}
\usepackage{tabularx}
\usepackage{graphicx}
\usepackage{float}
\usepackage{hyperref}
\hypersetup{
    colorlinks,
    citecolor=blue,
    filecolor=black,
    linkcolor=red,
    urlcolor=blue
}

%% Comments

\usepackage{color}

%\newif\ifcomments\commentstrue %displays comments
\newif\ifcomments\commentsfalse %so that comments do not display

\ifcomments
\newcommand{\authornote}[3]{\textcolor{#1}{[#3 ---#2]}}
\newcommand{\todo}[1]{\textcolor{red}{[TODO: #1]}}
\else
\newcommand{\authornote}[3]{}
\newcommand{\todo}[1]{}
\fi

\newcommand{\wss}[1]{\authornote{blue}{SS}{#1}} 
\newcommand{\plt}[1]{\authornote{magenta}{TPLT}{#1}} %For explanation of the template
\newcommand{\an}[1]{\authornote{cyan}{Author}{#1}}

%% Common Parts

\newcommand{\progname}{Mechatronics Engineering} % PUT YOUR PROGRAM NAME HERE
\newcommand{\authname}{Team 25, Preliminary
\\ Ahmed Nazir, nazira1
\\ Stephen Oh, ohs9
\\ Muhanad Sada, sadam
\\ Tioluwalayomi Babayeju, babayejt} % AUTHOR NAMES                  

\usepackage{hyperref}
    \hypersetup{colorlinks=true, linkcolor=blue, citecolor=blue, filecolor=blue,
                urlcolor=blue, unicode=false}
    \urlstyle{same}
                                


\newcounter{acnum}
\newcommand{\actheacnum}{AC\theacnum}
\newcommand{\acref}[1]{AC\ref{#1}}

\newcounter{ucnum}
\newcommand{\uctheucnum}{UC\theucnum}
\newcommand{\uref}[1]{UC\ref{#1}}

\newcounter{mnum}
\newcommand{\mthemnum}{M\themnum}
\newcommand{\mref}[1]{M\ref{#1}}

\begin{document}

\title{Module Guide for \progname{}} 
\author{\authname}
\date{\today}

\maketitle

\pagenumbering{roman}

\section{Revision History}

\begin{tabularx}{\textwidth}{p{3cm}p{2cm}X}
\toprule {\bf Date} & {\bf Version} & {\bf Notes}\\
\midrule
2023/01/18 & 1.0 & Final Version\\
\bottomrule
\end{tabularx}

\newpage

\section{Reference Material}

This section records information for easy reference.

\subsection{Abbreviations and Acronyms}

\renewcommand{\arraystretch}{1.2}
\begin{tabular}{l l} 
  \toprule		
  \textbf{symbol} & \textbf{description}\\
  \midrule 
  AC & Anticipated Change\\
  M & Module \\
  MG & Module Guide \\
  MIS & Module Interface Specifications \\ 
  OS & Operating System \\
  FR & Functional Requirement\\
  SRS & Software Requirements Specification\\
  UC & Unlikely Change \\
  \wss{etc.} & \wss{...}\\
  \bottomrule
\end{tabular}\\

\newpage

\tableofcontents

\listoftables


\newpage

\pagenumbering{arabic}

\section{Introduction}

The following document explains the Module Guide for our Capstone project. Complementary documents include the System Requirement Specifications, (SRS), and the Module Interface Specifications, (MIS). 

The rest of the document is organized as follows. Section
\ref{SecChange} lists the anticipated and unlikely changes of the software
requirements. Section \ref{SecMH} summarizes the module decomposition that
was constructed according to the likely changes. Section \ref{SecMD} gives a detailed description of the modules. Section \ref{SecTM} includes two traceability matrices. One checks
the completeness of the design against the requirements provided in the SRS. The
other shows the relation between anticipated changes and the modules.

\section{Anticipated and Unlikely Changes} \label{SecChange}

This section lists possible changes to the system. According to the likeliness
of the change, the possible changes are classified into two
categories. Anticipated changes are listed in Section \ref{SecAchange}, and
unlikely changes are listed in Section \ref{SecUchange}.

\subsection{Anticipated Changes} \label{SecAchange}

Anticipated changes are the source of the information that is to be hidden
inside the modules. Ideally, changing one of the anticipated changes will only
require changing the one module that hides the associated decision. The approach
adapted here is called design for
change.

\begin{description}
\item[\refstepcounter{acnum} \actheacnum \label{acHardware}:] The color scheme of the GUI which means a change in the stylesheet.
\item[\refstepcounter{acnum} \actheacnum \label{acInput}:] The layout of the pages and buttons in the GUI which means a change in the ui files and animation modules.
\item[\refstepcounter{acnum} \actheacnum \label{acGuiInfo}:] Information that is displayed on each GUI page.
\item[\refstepcounter{acnum} \actheacnum \label{acDbStoring}:] The method of storing information in the database to accommodate for large amounts of data.
\item[\refstepcounter{acnum} \actheacnum \label{acAccountFields}:] The fields required for creating an account.
\item[\refstepcounter{acnum} \actheacnum \label{acDataStructures}:] The data structures for storing and manipulating data inside the software.
\item[\refstepcounter{acnum} \actheacnum \label{acBackendFrontend}:] The method of decomposing front and back-end functionality of GUI.
\end{description}

\subsection{Unlikely Changes} \label{SecUchange}

The module design should be as general as possible. However, a general system is
more complex. Sometimes this complexity is not necessary. Fixing some design
decisions at the system architecture stage can simplify the software design. If
these decision should later need to be changed, then many parts of the design
will potentially need to be modified. Hence, it is not intended that these
decisions will be changed.

\begin{description}
\item[\refstepcounter{ucnum} \uctheucnum \label{ucIO}:] Input device will be a keyboard and output device will be a screen. 
\item[\refstepcounter{ucnum} \uctheucnum \label{ucQt}:] The Qt library for creating the GUI
\item[\refstepcounter{ucnum} \uctheucnum \label{ucHardware}:] The hardware the GUI is running on
\item[\refstepcounter{ucnum} \uctheucnum \label{ucPython}:] The programming language (python) for the GUI
\item[\refstepcounter{ucnum} \uctheucnum \label{ucStoringUser}:] Storing test and user information in a database
\item[\refstepcounter{ucnum} \uctheucnum \label{ucAzure}:] Hosting the database on an Azure SQL Server
\item[\refstepcounter{ucnum} \uctheucnum \label{ucPowerBi}:] Visually displaying stored test information in Power BI
\end{description}

\section{Module Hierarchy} \label{SecMH}

This section provides an overview of the module design. Modules are summarized
in a hierarchy decomposed by secrets in Table \ref{TblMH}. The modules listed
below, which are leaves in the hierarchy tree, are the modules that will
actually be implemented.

\begin{description}
\item [\refstepcounter{mnum} \mthemnum \label{Arduino}:] mainArduino.ino
\item [\refstepcounter{mnum} \mthemnum \label{ESP8266}:] mainESP8266.ino
\item [\refstepcounter{mnum} \mthemnum \label{ui_main}:] ui\_main.py
\item [\refstepcounter{mnum} \mthemnum \label{ui_functions}:] ui\_functions.py
\item [\refstepcounter{mnum} \mthemnum \label{main}:] main.py
\item [\refstepcounter{mnum} \mthemnum \label{resource_rc}:] resource\_rc.py

\end{description}


\begin{table}[h!]
\centering
\begin{tabular}{p{0.3\textwidth} p{0.6\textwidth}}
\toprule
\textbf{Level 1} & \textbf{Level 2}\\
\midrule

\multirow{2}{0.3\textwidth}{Hardware-Hiding Module} & mainArduino.ino \\
& mainESP8266.ino \\
\midrule

\multirow{3}{0.3\textwidth}{Behaviour-Hiding Module} & ui\_main.py\\
& ui\_functions.py\\
& main.py\\
\midrule

{Software Decision Module} & {resource\_rc.py}\\
\bottomrule

\end{tabular}
\caption{Module Hierarchy}
\label{TblMH}
\end{table}

\section{Module Decomposition} \label{SecMD}


\subsection{Hardware Hiding Modules}

\subsubsection{mainArduino.ino (\mref{Arduino})}
\begin{description}
\item[Services:]Converts the input data taken from sensors into the data that is converted to data used by the UI.
\item[Implemented By:] [Arduino]
\end{description}

\subsubsection{mainESP8266.ino (\mref{ESP8266})}
\begin{description}
\item[Services:] An Arduino module establishes a Wi-Fi connection between the Arduino and PC to send data from the sensors and the Arduino to the PC.
\item[Implemented By:] [Arduino, ESP8266]
\end{description}

\subsection{Behaviour-Hiding Module}

\subsubsection{ui\_main.py (\mref{ui_main})}

\begin{description}
\item[Services:] A PyQt designer generated module that sets up the application's window and design.
\item[Implemented By:] [PyQt]
\end{description}

\subsubsection{ui\_functions.py (\mref{ui_functions})}
\begin{description}
\item[Services:] A python module that imports all the necessary libraries for backend, contains class for UI functions and creates the connection to the database.
\item[Implemented By:] [PyQt]
\end{description}

\subsubsection{ui\_functions.py (\mref{main})}
\begin{description}
\item[Services:] A python module that imports the backend functions used in ui\_funtions.py and does the frontend setup for the GUI.
\item[Implemented By:] [PyQt]
\end{description}

\subsection{Software Decision Module}

\subsubsection{resource\_rc.py (\mref{resource_rc})}

\begin{description}

  \item[Services:] A python module generated by PyQt resource compiler and uses all the local resources to display images during runtime.
  \item[Implemented By:] [PyQt] 
  \end{description}

\newpage
\section{Traceability Matrix} \label{SecTM}

This section shows two traceability matrices: between the modules and the
functional requirements from our SRS document and between the modules and the anticipated changes.


% the table should use mref, the requirements should be named, use something
% like fref
\begin{table}[H]
\centering
\begin{tabular}{p{0.2\textwidth} p{0.6\textwidth}}
\toprule
\textbf{Req.} & \textbf{Modules}\\
\midrule
FR1 & \mref{Arduino}\\
FR2 & \mref{Arduino}, \mref{ESP8266}\\
FR3 & \mref{Arduino}, \mref{ESP8266}\\
FR4 & \mref{Arduino}, \mref{ESP8266}\\
FR5 & \mref{Arduino}\\
FR6 & \mref{Arduino}\\
FR7 & \mref{Arduino}, \mref{ui_functions}\\
FR8 & \mref{ui_functions}\\
FR12 & \mref{Arduino}, \mref{ESP8266}\\
FR13 & \mref{Arduino}, \mref{ui_main}, \mref{ui_functions}, \mref{main}\\
FR15 & \mref{Arduino}, \mref{ESP8266}, \mref{ui_main}, \mref{ui_functions}, \mref{main}\\
FR16 & \mref{Arduino}, \mref{ui_functions}, \mref{main}\\
FR18 & \mref{ui_main}, \mref{ui_functions}, \mref{main}\\
FR19 & \mref{ui_functions}, \mref{main}\\
FR20 & \mref{ui_main}, \mref{ui_functions}, \mref{main}\\
FR21 & \mref{ui_functions}, \mref{main}\\
FR22 & \mref{ui_functions}, \mref{main}\\
FR23 & \mref{Arduino}, \mref{ui_main}, \mref{ui_functions}\\
\bottomrule
\end{tabular}
\caption{Trace Between Requirements and Modules}
\label{TblRT}
\end{table}

\begin{table}[H]
\centering
\begin{tabular}{p{0.2\textwidth} p{0.6\textwidth}}
\toprule
\textbf{AC} & \textbf{Modules}\\
\midrule
\acref{acHardware} & \mref{ui_main}, \mref{ui_functions}, \mref{main}\\
\acref{acInput} & \mref{ui_main}, \mref{ui_functions}, \mref{main}\\
\acref{acGuiInfo} & \mref{main}\\
\acref{acDbStoring} & \mref{ui_main}, \mref{ui_functions}\\
\acref{acAccountFields} & \mref{ui_functions}\\
\acref{acDataStructures} & \mref{ui_functions}\\
\acref{acBackendFrontend} & \mref{ui_functions}, \mref{main}\\

\bottomrule
\end{tabular}
\caption{Trace Between Anticipated Changes and Modules}
\label{TblACT}
\end{table}

\newpage{}

\end{document}