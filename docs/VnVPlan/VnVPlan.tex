\documentclass[12pt, titlepage]{article}

\usepackage{booktabs}
\usepackage{tabularx}
\usepackage{hyperref}
\usepackage{float}
\hypersetup{
    colorlinks,
    citecolor=blue,
    filecolor=black,
    linkcolor=red,
    urlcolor=blue
}
\usepackage[round]{natbib}

%% Comments

\usepackage{color}

%\newif\ifcomments\commentstrue %displays comments
\newif\ifcomments\commentsfalse %so that comments do not display

\ifcomments
\newcommand{\authornote}[3]{\textcolor{#1}{[#3 ---#2]}}
\newcommand{\todo}[1]{\textcolor{red}{[TODO: #1]}}
\else
\newcommand{\authornote}[3]{}
\newcommand{\todo}[1]{}
\fi

\newcommand{\wss}[1]{\authornote{blue}{SS}{#1}} 
\newcommand{\plt}[1]{\authornote{magenta}{TPLT}{#1}} %For explanation of the template
\newcommand{\an}[1]{\authornote{cyan}{Author}{#1}}

%% Common Parts

\newcommand{\progname}{Mechatronics Engineering} % PUT YOUR PROGRAM NAME HERE
\newcommand{\authname}{Team 25, Preliminary
\\ Ahmed Nazir, nazira1
\\ Stephen Oh, ohs9
\\ Muhanad Sada, sadam
\\ Tioluwalayomi Babayeju, babayejt} % AUTHOR NAMES                  

\usepackage{hyperref}
    \hypersetup{colorlinks=true, linkcolor=blue, citecolor=blue, filecolor=blue,
                urlcolor=blue, unicode=false}
    \urlstyle{same}
                                


\begin{document}

\title{Project Title: System Verification and Validation Plan for \progname{}} 
\author{\authname}
\date{\today}
	
\maketitle

\pagenumbering{roman}

\section{Revision History}

\begin{tabularx}{\textwidth}{p{3cm}p{2cm}X}
\toprule {\bf Date} & {\bf Developer} & {\bf Notes}\\
\midrule
October 30 & Stephen & Added General Information and Plan\\
Date 2 & 1.1 & Notes\\
\bottomrule
\end{tabularx}

\newpage

\tableofcontents

\listoftables
\wss{Remove this section if it isn't needed}

\listoffigures
\wss{Remove this section if it isn't needed}

\newpage

\section{Symbols, Abbreviations and Acronyms}

\renewcommand{\arraystretch}{1.2}
\begin{tabular}{l l} 
  \toprule		
  \textbf{symbol} & \textbf{description}\\
  \midrule 
  T & Test\\
  \bottomrule
\end{tabular}\\

\wss{symbols, abbreviations or acronyms -- you can simply reference the SRS
  \citep{SRS} tables, if appropriate}

\newpage

\pagenumbering{arabic}

This document ... \wss{provide an introductory blurb and roadmap of the
  Verification and Validation plan}

\section{General Information}

\subsection{Summary}

\wss{Say what software is being tested.  Give its name and a brief overview of
  its general functions.}

  Formulate consists of four subsystems, one hardware subsystem and three software subsystems, which interact to provide the user with a testing device designed to eliminate automatable processes in common testing procedures.\\

  A physical data collection device is the hardware subsystem used as the first point of contact with the measured quantity through a sensor. The sensor obtains physical quantities for the device to buffer, before sending the data to a desktop application for user verification.\\

  The user has the ability to view the data collected by the physical device after a completed test using a desktop application software subsystem. The desktop application enables the user to either accept the test results to then store a collection of data from a test to a database, or reject the test and prevent the test from being stored to a database.\\

  Accepted test data sent from the desktop application aggregates and saves verified test results to a database software subsystem. Users can query the database to obtain common statistics in test data and generate new or obscure relationships by leveraging database language capabilities.\\

  A final dashboard software subsystem then queries the database to visualize key performance indicators on the test data collected and stored in the database.\newpage





\subsection{Objectives}

\wss{State what is intended to be accomplished.  The objective will be around
  the qualities that are most important for your project.  You might have
  something like: ``build confidence in the software correctness,''
  ``demonstrate adequate usability.'' etc.  You won't list all of the qualities,
  just those that are most important.}

  The objective of the system Verification and Validation (VnV) plan for Formulate is to ensure the intended project qualities are present.\\

  Ease in user understanding is a quality Formulate will achieve to support the system's usability.  Specifically, ease in user understanding of each subsystem's function and how subsystems interface will be key qualities of the overall project.\\

  The system will also demonstrate the quality of adequate portability and physical robustness to support system maintainability, portability, and operationality.\\



\subsection{Relevant Documentation}

\wss{Reference relevant documentation.  This will definitely include your SRS
  and your other project documents (MG, MIS, etc).  You can include these even
  before they are written, since by the time the project is done, they will be
  written.}

  Talk about how this document draws from the system requirements gathered during software requirements specification (SRS) and hazard analysis.\\

  This document references a variety of requirements generated during the Software Requirements Specification (SRS) process and the Hazard Analysis (HA) process for the Formulate system. \\


\citet{SRS}
\newpage
\section{Plan}

\wss{Introduce this section.   You can provide a roadmap of the sections to
  come.}
  
  \subsection{Roadmap}

  The intention of testing for Formulate is to generate confidence that the project meets qualities relating to usability, maintainability, portability, operationality, and safety set out as requirements in SRS and HA documentation. Through sets of unit and system tests that will prove if the system has met the above requirements, Formulate will understand if the project has achieved the desired qualities.\\

  Specifically, requirements that are functional, non-functional, and safety-security related from the SRS and HA documents will be referenced in the Plan, System Test Description, and Unit Test Description sections of this document.\\

\subsection{Verification and Validation Team}

\wss{You, your classmates and the course instructor.  Maybe your supervisor.
  You shoud do more than list names.  You should say what each person's role is
  for the project.  A table is a good way to summarize this information.}

  \begin{table}[H]
    \centering
    \begin{tabular}{|p{3cm}|p{4cm}|p{7cm}|}
    \hline
    \multicolumn{1}{|c|}{\textbf{Name}} & \multicolumn{1}{|c|}{\textbf{Role}} & \multicolumn{1}{|c|}{\textbf{Explanation}}
    \\ \hline
    Stephen
    & Desktop Application Tester
    & VnV for software application design and integration with hardware and database
    \newline                                
    \\ \hline
  
    Ahmed                              
    & Hardware Device Tester
    & VnV for embedded program design, chassis design, and integration with desktop application
    \newline                                
    \\ \hline
  
    Muhanad                          
    & Database and Visualization Application Tester
    & VnV for database design and integration with desktop application and dashboard
    \newline                                
    \\ \hline
  
    Tioluwalayomi                                
    & Dashboard Application Tester
    & VnV for dashboard application design and integration with database
    \newline                            
    \\ \hline

    Timofey                                
    & Project and Course Teaching Assistant
    & Detailed low level feedback on planned VnV tests
    \newline                            
    \\ \hline
  
    Dr. Smith                                
    & Course Instructor
    & General high level feedback on planned VnV tests 
    \newline                            
    \\ \hline
  
    \end{tabular}
  \end{table}
  \newpage

\subsection{SRS Verification Plan}

\wss{List any approaches you intend to use for SRS verification.  This may just
  be ad hoc feedback from reviewers, like your classmates, or you may have
  something more rigorous/systematic in mind..}

\wss{Remember you have an SRS checklist}

SRS Verification will be composed of two approaches to verify that functional and non-functional requirements are met. The first approach is engaging in read through's of the SRS document each month. Individual member progress will be evaluated against the relevenat sections(s) of the SRS document to ensure system  development is on track to meet the requirements. The second approach is evaluating issues created by classmates on GitHub and incorporating their concerns and suggestions as seen fit.\\

Stephen will lead the group wide discussion for SRS verification activities on the first Tuesday of each month.\\

\subsection{Design Verification Plan}



\wss{Plans for design verification}

\wss{The review will include reviews by your classmates}

\wss{Remember you have MG and MIS checklists}

\subsection{Implementation Verification Plan}

\wss{You should at least point to the tests listed in this document and the unit
  testing plan.}

\wss{In this section you would also give any details of any plans for static verification of
  the implementation.  Potential techniques include code walkthroughs, code
  inspection, static analyzers, etc.}

\subsection{Automated Testing and Verification Tools}

\wss{What tools are you using for automated testing.  Likely a unit testing
  framework and maybe a profiling tool, like ValGrind.  Other possible tools
  include a static analyzer, make, continuous integration tools, test coverage
  tools, etc.  Explain your plans for summarizing code coverage metrics.
  Linters are another important class of tools.  For the programming language
  you select, you should look at the available linters.  There may also be tools
  that verify that coding standards have been respected, like flake9 for
  Python.}

\wss{The details of this section will likely evolve as you get closer to the
  implementation.}

\subsection{Software Validation Plan}

\wss{If there is any external data that can be used for validation, you should
  point to it here.  If there are no plans for validation, you should state that
  here.}

\section{System Test Description}
	
\subsection{Tests for Functional Requirements}

\wss{Subsets of the tests may be in related, so this section is divided into
  different areas.  If there are no identifiable subsets for the tests, this
  level of document structure can be removed.}

\wss{Include a blurb here to explain why the subsections below
  cover the requirements.  References to the SRS would be good.}

\subsubsection{Area of Testing1}

\wss{It would be nice to have a blurb here to explain why the subsections below
  cover the requirements.  References to the SRS would be good.  If a section
  covers tests for input constraints, you should reference the data constraints
  table in the SRS.}
		
\paragraph{Title for Test}

\begin{enumerate}

\item{test-id1\\}

Control: Manual versus Automatic
					
Initial State: 
					
Input: 
					
Output: \wss{The expected result for the given inputs}

Test Case Derivation: \wss{Justify the expected value given in the Output field}
					
How test will be performed: 
					
\item{test-id2\\}

Control: Manual versus Automatic
					
Initial State: 
					
Input: 
					
Output: \wss{The expected result for the given inputs}

Test Case Derivation: \wss{Justify the expected value given in the Output field}

How test will be performed: 

\end{enumerate}

\subsubsection{Area of Testing2}

...

\subsection{Tests for Nonfunctional Requirements}

\wss{The nonfunctional requirements for accuracy will likely just reference the
  appropriate functional tests from above.  The test cases should mention
  reporting the relative error for these tests.}

\wss{Tests related to usability could include conducting a usability test and
  survey.}

\subsubsection{Performance}
		
\paragraph{Operational in physical environment}

\begin{enumerate}

\item{Operational in physical enviornment\\}

Type: Dynamic, Manual
					
Initial State: Device is on and mounted to the device, has connected to the application and is waiting to start measuring.
					
Input/Condition: Vehicle's motor starts and values start to get picked up by device
					
Output/Result: Device is operational and stays physically intact in all types of weather and at 20\% greater than threshold values.
					
How test will be performed: The device will be tested outdoors under various weather conditions including rain, windy, etc.
The device will also be tested in temperature and vibration conditions that are above threshold values. This will be performed by placing the device in a hot environment
and vigoursly shaking it while being on a stationary mount.
					
\item{Viewing live data\\}

Type: Dynamic, Manual
					
Initial State: Device is on and mounted to the device, has connected to the application and is waiting to start measuring.
					
Input: Vehicle's motor starts and values start to get picked up by device
					
Output/Result: Data latency should be less than 30 seconds to simulate viewing live data.
					
How test will be performed: The amount of time for data to start being viewable on the application will be inspected to be less than 30 seconds.
The application will also be inspected to ensure that data is smooth and not lagging while measurements are being performed.

\item{Modularity and Maintainability\\}

Type: Dynamic, Manual
					
Initial State: Device is on measuring and sending values to the application, and connection to database has been verified
					
Input: Either the device, application, or database is disconnected or turned off
					
Output: The other two components are still functional even though communication between them is broken.
					
How test will be performed: While device, application, and database are fully functional and communicating successfully, different combinations of either one or two components
will be turned off. The other component(s) will be inspected to ensure that they are operational and indicating that the other component(s) are disconnected.

\item{test-id2\\}

Type: Functional, Dynamic, Manual, Static etc.
					
Initial State: 
					
Input: 
					
Output: 
					
How test will be performed: 

\item{test-id2\\}

Type: Functional, Dynamic, Manual, Static etc.
					
Initial State: 
					
Input: 
					
Output: 
					
How test will be performed: 

\end{enumerate}

\subsubsection{Usability}

\begin{enumerate}

\item{Mounting hardware and starting measurements\\}

Type: Dynamic, Manual
          
Initial State: Device is turned off and nothing is connected, only the application is loaded on to the computer
          
Input: Users will be asked to setup device and start taking measurements, rate setup process using a survey
          
Output: Time for setup and data to appear on the application should be less than 5 minutes and 
          
How test will be performed: A test group will be educated on the setup and connection of the device, then they will attempt to do that process. 
Each person will be timed and compared to the 5 minute threshold. In addition, they will be given a survey to rate the setup process on a scale from 
1 to 5 the following categories: ease of use, need for assistance,  

\item{test-id2\\}

Type: Dynamic, Manual
					
Initial State: Device is given to McMaster's Formula E team to use
					
Input: Using a survey, Formula E members will compare their current testing process to the Formulate process
					
Output: All users need to select Formulate in at least 2 of the 3 categories
					
How test will be performed: Formula E members will select which process is preferred in the following categories: speed, data collection, ease of use

\item{test-id2\\}

Type: Dynamic, Manual
					
Initial State: 
					
Input: 
					
Output: 
					
How test will be performed: 

\end{enumerate}

\subsection{Traceability Between Test Cases and Requirements}

\wss{Provide a table that shows which test cases are supporting which
  requirements.}

\section{Unit Test Description}

\wss{Reference your MIS and explain your overall philosophy for test case
  selection.}  
\wss{This section should not be filled in until after the MIS has
  been completed.}

\subsection{Unit Testing Scope}

\wss{What modules are outside of the scope.  If there are modules that are
  developed by someone else, then you would say here if you aren't planning on
  verifying them.  There may also be modules that are part of your software, but
  have a lower priority for verification than others.  If this is the case,
  explain your rationale for the ranking of module importance.}

\subsection{Tests for Functional Requirements}

\wss{Most of the verification will be through automated unit testing.  If
  appropriate specific modules can be verified by a non-testing based
  technique.  That can also be documented in this section.}

\subsubsection{Module 1}

\wss{Include a blurb here to explain why the subsections below cover the module.
  References to the MIS would be good.  You will want tests from a black box
  perspective and from a white box perspective.  Explain to the reader how the
  tests were selected.}

\begin{enumerate}

\item{test-id1\\}

Type: \wss{Functional, Dynamic, Manual, Automatic, Static etc. Most will
  be automatic}
					
Initial State: 
					
Input: 
					
Output: \wss{The expected result for the given inputs}

Test Case Derivation: \wss{Justify the expected value given in the Output field}

How test will be performed: 
					
\item{test-id2\\}

Type: \wss{Functional, Dynamic, Manual, Automatic, Static etc. Most will
  be automatic}
					
Initial State: 
					
Input: 
					
Output: \wss{The expected result for the given inputs}

Test Case Derivation: \wss{Justify the expected value given in the Output field}

How test will be performed: 

\item{...\\}
    
\end{enumerate}

\subsubsection{Module 2}

...

\subsection{Tests for Nonfunctional Requirements}

\wss{If there is a module that needs to be independently assessed for
  performance, those test cases can go here.  In some projects, planning for
  nonfunctional tests of units will not be that relevant.}

\wss{These tests may involve collecting performance data from previously
  mentioned functional tests.}

\subsubsection{Module ?}
		
\begin{enumerate}

\item{test-id1\\}

Type: \wss{Functional, Dynamic, Manual, Automatic, Static etc. Most will
  be automatic}
					
Initial State: 
					
Input/Condition: 
					
Output/Result: 
					
How test will be performed: 
					
\item{test-id2\\}

Type: Functional, Dynamic, Manual, Static etc.
					
Initial State: 
					
Input: 
					
Output: 
					
How test will be performed: 

\end{enumerate}

\subsubsection{Module ?}

...

\subsection{Traceability Between Test Cases and Modules}

\wss{Provide evidence that all of the modules have been considered.}
				
\bibliographystyle{plainnat}

\bibliography{../../refs/References}

\newpage

\section{Appendix}

This is where you can place additional information.

\subsection{Symbolic Parameters}

The definition of the test cases will call for SYMBOLIC\_CONSTANTS.
Their values are defined in this section for easy maintenance.

\subsection{Usability Survey Questions?}

\wss{This is a section that would be appropriate for some projects.}

\end{document}