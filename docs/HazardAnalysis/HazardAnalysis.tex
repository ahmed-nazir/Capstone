\documentclass{article}

\usepackage{booktabs}
\usepackage{tabularx}
\usepackage{hyperref}
\usepackage{colortbl}
\usepackage{fullpage}
\usepackage{longtable}
\newcounter{fmeanum}
\newcounter{fmeanumDA}
\newcounter{fmeanumD}
\newcounter{fmeanumDAW}
\newcounter{reqnum} %Requirement Number

\hypersetup{
    colorlinks=true,       % false: boxed links; true: colored links
    linkcolor=red,          % color of internal links (change box color with linkbordercolor)
    citecolor=green,        % color of links to bibliography
    filecolor=magenta,      % color of file links
    urlcolor=cyan           % color of external links
}

\title{Hazard Analysis\\\progname}

\author{\authname}

\date{}

%% Comments

\usepackage{color}

%\newif\ifcomments\commentstrue %displays comments
\newif\ifcomments\commentsfalse %so that comments do not display

\ifcomments
\newcommand{\authornote}[3]{\textcolor{#1}{[#3 ---#2]}}
\newcommand{\todo}[1]{\textcolor{red}{[TODO: #1]}}
\else
\newcommand{\authornote}[3]{}
\newcommand{\todo}[1]{}
\fi

\newcommand{\wss}[1]{\authornote{blue}{SS}{#1}} 
\newcommand{\plt}[1]{\authornote{magenta}{TPLT}{#1}} %For explanation of the template
\newcommand{\an}[1]{\authornote{cyan}{Author}{#1}}

%% Common Parts

\newcommand{\progname}{Mechatronics Engineering} % PUT YOUR PROGRAM NAME HERE
\newcommand{\authname}{Team 25, Preliminary
\\ Ahmed Nazir, nazira1
\\ Stephen Oh, ohs9
\\ Muhanad Sada, sadam
\\ Tioluwalayomi Babayeju, babayejt} % AUTHOR NAMES                  

\usepackage{hyperref}
    \hypersetup{colorlinks=true, linkcolor=blue, citecolor=blue, filecolor=blue,
                urlcolor=blue, unicode=false}
    \urlstyle{same}
                                


\begin{document}

\maketitle
\thispagestyle{empty}

~\newpage

\pagenumbering{roman}

\begin{table}[hp]
\caption{Revision History} \label{TblRevisionHistory}
\begin{tabularx}{\textwidth}{llX}
\toprule
\textbf{Date} & \textbf{Developer(s)} & \textbf{Change}\\
\midrule
10/12/2022 & Ahmed & Added FMEA analysis\\
10/13/2022 & Stephen & Added Introduction and Scope\\
10/13/2022 & Muhanad & Added Critical Assumptions and Safety Requirements\\
10/14/2022 & Tioluwalayomi & Added System Boundaries and Roadmap\\
10/19/2022 & All & General edits\\
\bottomrule
\end{tabularx}
\end{table}

~\newpage

\tableofcontents



~\newpage

\pagenumbering{arabic}

\wss{You are free to modify this template.}

\section{Introduction}

A hazard is the combination of a system property with an environmental condition that can cause harm to the intended user.\\

Hazard analysis is a critical consideration in the design of all systems. When done correctly, hazards to the end user are identified and can be mitigated or eliminated completely. While it is not possible to guarantee the safety of a system, applying hazard analysis methods is a neccesary step in supporting the safety of the system. \\

Formulate's area of work combines hardware and software sub-systems and as a result, requires hazard analysis to obtain a comprehensive understanding of the overall system. \\

\wss{You can include your definition of what a hazard is here.}

\section{Scope and Purpose of Hazard Analysis}

In this document, Formulate details the hazards a user can experience through the Failure Mode and Effect Analysis method. As a result, the group systematically outlined the hazards and measures that were considered to mitigate or eliminate the hazard.

\section{System Boundaries and Components}

The device that is referred to in this document is made up of 5 major components that hazard and failure analysis would have to be done for:

\begin{enumerate}
\item Hardware
\item Desktop Application
\item Database
\item Data Analytics Website 
\end{enumerate}

Each component has there own system boundaries based on the software and hardware we use. Since this is the case we will have to design the system based on the type of the test that is required to be performed by the MAC Formula Electric Team. An example of this would be if the client had to test the car under specific conditions, we would choose hardware components that are within that temperature range to ensure ensure data is read correctly.

\section{Critical Assumptions}

\begin{itemize}
	\item The user understands the safety precautions of operating and working with a Formula Electric vehicle
	\item The user has a basic understanding of handling electrical and mechanical components
	\item The user aims to always correctly operate the testing device
\end{itemize}

\wss{These assumptions that are made about the software or system.  You should
minimize the number of assumptions that remove potential hazards.  For instance,
you could assume a part will never fail, but it is generally better to include
this potential failure mode.}

\newpage
\section{Failure Mode and Effect Analysis}
\begin{longtable}{| p{0.12\textwidth} | p{0.04\textwidth}| p{0.15\textwidth}| p{0.20\textwidth}| p{0.20\textwidth}| p{0.18\textwidth}|}
    \hline
    \rowcolor[gray]{0.9}
    \textbf{Component} 
    & \textbf{Ref}
    & \textbf{Failure Mode}
    & \textbf{Effects of Failure} 
    & \textbf{Cause of Failure}
    & \textbf{Recommended Actions} \\
    %Component & Ref & Failure Mode & Effects of Failure & Cause of Failure & Recommended Actions\\
    \hline 
    Hardware & H1.\refstepcounter{fmeanum}\thefmeanum
    & Connection Failure
    & Test data is not captured by our PC
    &   $\bullet$ The Wi-Fi module is broken \newline
        $\bullet$ USB Device is not connected to PC \newline
        $\bullet$ Device is not connected to Wi-Fi a network \newline 
    & Using the LCD display show the systems connectivity \\
    \cline{2-6}
        & H1.\refstepcounter{fmeanum}\thefmeanum
    & System does not have power
    & Device is off and not operational
    &   $\bullet$ Battery died \newline
        $\bullet$ Power cables are disconnected \newline
        $\bullet$ Too much current is drawn from Arduino \newline 
    & $\bullet$ Add a battery indicator to the screen to alert the user if the battery is low \newline
      $\bullet$ Make the sensors get their power directly from the power source and not the arduino \newline \\

    \cline{2-6}
        & H1.\refstepcounter{fmeanum}\thefmeanum
    & Hardware falls off the mount
    &   $\bullet$ Hardware device breaks/gets damaged \newline
        $\bullet$ Sensors capture incorrect data \newline
        $\bullet$ Potential injury to those in vehicle \newline
    &  $\bullet$ User didn’t affix Hardware properly \newline
       $\bullet$ Mounting mechanism failed \newline
    & The mounting mechanism should give the user feedback when the device is mounted correctly\\
    \cline{2-6}
        & H1.\refstepcounter{fmeanum}\thefmeanum
    & Display turns off
    & Cannot view the status of the device
    &   $\bullet$ LCD display failure \newline
        $\bullet$ LCD is improperly connected \newline
        $\bullet$ Arduino is drawing too much current \newline 
    &Replace the display and ensure the display is connected properly prior to installation\\
    
    \endfirsthead
    \cline{2-6}
    
        & H1.\refstepcounter{fmeanum}\thefmeanum
    & Threshold alert not displaying
    & User will not be notified when testing components go out of their operating range
    & $\bullet$ Sensor failure \newline
      $\bullet$ Refer to H1.4 \newline
      $\bullet$ Threshold not set up by user in the Desktop App \newline
    &Warn user that threshold value is not set in the Desktop App\\
    
    \hline
    Desktop \space Application & H2.\refstepcounter{fmeanumDA}\thefmeanumDA
    & Application cannot read hardware device output
    & Refer to H1.1
    & $\bullet$ Refer to H1.1 \newline
      $\bullet$ COM Port is being used by another application \newline
    &Ensure that the COM port is open for communication\\

    \cline{2-6}
     & H2.\refstepcounter{fmeanumDA}\thefmeanumDA
    & Data from the hardware device is lost
    & Test results will all be lost
    & $\bullet$ Application suddenly closes during test \newline
      $\bullet$ Hardware device disconnects from PC \newline
    &Store last test data into local storage\\

    \cline{2-6}
     & H2.\refstepcounter{fmeanumDA}\thefmeanumDA
    & Cannot view live data
    & User will not be able to see live data during test runs
    & $\bullet$ Refer to H2.1\newline
      $\bullet$ Refer to H1.1 \newline
    & Device will default to record the entire test run and save the test data into local storage\\

    \cline{2-6}
     & H2.\refstepcounter{fmeanumDA}\thefmeanumDA
    & Data cannot be sent to database
    & Test results will not be saved to the database and will not be viewable in the analytics platform
    & $\bullet$ Database failure \newline
      $\bullet$ Connection failure between Desktop App and Database  \newline
      $\bullet$ PC not connected to the internet \newline
    &Validate connection between desktop app and database to ensure data will be sent correctly\\

    \hline

    Database & H3.\refstepcounter{fmeanumD}\thefmeanumD
    & Database Overload
    & The database is getting overloaded with data causing it to crash or freeze
    & User submits too much data within a very short time period
    & Add a cool down timer after the user submits the data to the database so they will not be able to spam it constantly\\
    \hline

    Data \newline Analytics Website & H4.\refstepcounter{fmeanumDAW}\thefmeanumDAW
    & User cannot login
    & User will not have access to dashboard
    & $\bullet$ User does not have an account \newline
      $\bullet$ User’s credentials do not match \newline
    &\\
    \cline{2-6}

    & H4.\refstepcounter{fmeanumDAW}\thefmeanumDAW
    & User cannot view the dashboard
    & Users cannot view KPIs of tests
    & $\bullet$ User does not have required permissions \newline
    &\\
    \cline{2-3}\cline{5-6}

    & H4.\refstepcounter{fmeanumDAW}\thefmeanumDAW
    & Data not being displayed
    & 
    & $\bullet$ Database failure \newline
    $\bullet$ Authentication error \newline
    &\\
    \hline


    \end{longtable}
    


\wss{Include your FMEA table here}

\section{Safety and Security Requirements}

\begin{itemize}
  \item[SR \refstepcounter{reqnum}\thereqnum:] The device should validate a connection between sensors, device, and application before starting measurements.
    \begin{description} \item[Rationale] If the connections are not validated, then data might be lost as it never gets send to either the device, application and/or database.  \end{description}
    \begin{description} \item[Associated Hazards] H1.1, H2.1, H2.2, H2.3, H2.4  \end{description}
  
  \item[SR \refstepcounter{reqnum}\thereqnum:] The device should give a warning indicating that battery is low after a certain time amount of usage.
    \begin{description} \item[Rationale] If the battery dies during testing, measurement values will be lost as it is the power source of the device.  \end{description}
    \begin{description} \item[Associated Hazards] H1.2  \end{description}
  
  \item[SR \refstepcounter{reqnum}\thereqnum:] In order to send values to the database from the application, user must log into the application.
    \begin{description} \item[Rationale] This ensures that only testers have the ability to send and write values to the database.   \end{description}
    \begin{description} \item[Associated Hazards] H4.1, H4.2, H4.3 \end{description}
    
  \item[SR \refstepcounter{reqnum}\thereqnum:] Every user should have a login when accessing the data analytics website which would be the same login as the application's.
    \begin{description} \item[Rationale] This ensures that only testers have view access to values on the website.   \end{description}
    \begin{description} \item[Associated Hazards] H4.1, H4.2, H4.3 \end{description}
  
  \item[SR \refstepcounter{reqnum}\thereqnum:] The device should save a certain amount of values in case the connection to the application fails during testing.
    \begin{description} \item[Rationale] This ensures that the most recent data is not lost if there is a system failure.   \end{description}
    \begin{description} \item[Associated Hazards] H2.2  \end{description}
  
  \item[SR \refstepcounter{reqnum}\thereqnum:] The device should have an audible click to indicate that the device and sensors have mounted correctly.
    \begin{description} \item[Rationale] The audible click aids the user in mounting the hardware correctly to ensure that it does not fall off.   \end{description}
    \begin{description} \item[Associated Hazards] H1.3  \end{description}
\end{itemize}

\wss{Newly discovered requirements.  These should also be added to the SRS.  (A
rationale design process how and why to fake it.)}

\section{Roadmap}

All these requirements will try to be implemented during the course of the capstone. The safety and security requirements are important for a complete and safe testing device. As this  is the case most, but hopefully all of them will be implemented but if we can not implement all of them due to our time constraint the ones essential to the design will be implemented while the others that only improve user functionality will be added towards the end.

\end{document}