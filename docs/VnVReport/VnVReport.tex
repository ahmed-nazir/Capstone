\documentclass[12pt, titlepage]{article}

\usepackage{booktabs}
\usepackage{tabularx}
\usepackage{hyperref}
\hypersetup{
    colorlinks,
    citecolor=black,
    filecolor=black,
    linkcolor=red,
    urlcolor=blue
}
\usepackage[round]{natbib}

\input{../Comments}
%% Common Parts

\newcommand{\progname}{MECHTRON 4TB6} % PUT YOUR PROGRAM NAME HERE
\newcommand{\authname}{Team 25, Formulate
\\ Ahmed Nazir, nazira1
\\ Stephen Oh, ohs9
\\ Muhanad Sada, sadam
\\ Tioluwalayomi Babayeju, babayejt} % AUTHOR NAMES                  

\usepackage{hyperref}
    \hypersetup{colorlinks=true, linkcolor=blue, citecolor=blue, filecolor=blue,
                urlcolor=blue, unicode=false}
    \urlstyle{same}
                                


\begin{document}

\title{Verification and Validation Report: \progname} 
\author{\authname}
\date{\today}
	
\maketitle

\pagenumbering{roman}

\section{Revision History}

\begin{tabularx}{\textwidth}{p{3cm}p{2cm}X}
\toprule {\bf Date} & {\bf Version} & {\bf Notes}\\
\midrule
Date 1 & 1.0 & Notes\\
Date 2 & 1.1 & Notes\\
\bottomrule
\end{tabularx}

~\newpage

\section{Symbols, Abbreviations and Acronyms}

\renewcommand{\arraystretch}{1.2}
\begin{tabular}{l l} 
  \toprule		
  \textbf{symbol} & \textbf{description}\\
  \midrule 
  T & Test\\
  \bottomrule
\end{tabular}\\

\wss{symbols, abbreviations or acronyms -- you can reference the SRS tables if needed}

\newpage

\tableofcontents

\listoftables %if appropriate

\listoffigures %if appropriate

\newpage

\pagenumbering{arabic}

This document ...

\section{Functional Requirements Evaluation}
%sensor validation: Stephen
%device telemetry: Ahmed
%device hardware: Stephen
%desktop app: Mo
%data analytics: Tio
%database: Mo
\section{Nonfunctional Requirements Evaluation}
\subsection{Usability}
%Stephen
%Need a graph (that we generate) that supports data

\subsection{Performance}
%Tio
%Need a graph (that we generate) that supports data 

\subsection{Security}
%Tio

\section{Unit Testing}
%Desktop App Code: Mo
%Arduino Code: Ahmed

\section{Changes Due to Testing}
%Each person will have to write a section here. Each FR and NFR will have a paragraph here, so based on what was assigned earlier, we will need to write a section here.

\subsection{Functional Requirements}
%NOTE: Only keep requirements in here that were changed
%FR 1 :
%Point 1: Changes
%Point 2: Effect the change had on project

\subsection{Nonfunctional Requirements}

\wss{This section should highlight how feedback from the users and from 
the supervisor (when one exists) shaped the final product.  In particular 
the feedback from the Rev 0 demo to the supervisor (or to potential users) 
should be highlighted.}
		
\section{Trace to Requirements}
%VnV Plan has this table
		
\section{Trace to Modules}		
%Mapping unit tests to modules: MG has this table

\bibliographystyle{plainnat}
\bibliography{../../refs/References}

\newpage{}
\section*{Appendix --- Reflection}

The information in this section will be used to evaluate the team members on the
graduate attribute of Reflection.  Please answer the following question:

\begin{enumerate}
  \item In what ways was the Verification and Validation (VnV) Plan different
  from the activities that were actually conducted for VnV?  If there were
  differences, what changes required the modification in the plan?  Why did
  these changes occur?  Would you be able to anticipate these changes in future
  projects?  If there weren't any differences, how was your team able to clearly
  predict a feasible amount of effort and the right tasks needed to build the
  evidence that demonstrates the required quality?  (It is expected that most
  teams will have had to deviate from their original VnV Plan.)

  %Each person has 1 paragraph on stuff they worked that they can comment on. Ideally, 4 paragraphs total
\end{enumerate}

\end{document}