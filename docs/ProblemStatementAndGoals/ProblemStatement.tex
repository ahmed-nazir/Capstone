\documentclass[12pt]{article}

\usepackage{tabularx}
\usepackage{booktabs}
\usepackage{float}
\usepackage{fullpage}
\usepackage{longtable}


\title{Problem Statement and Goals\\\progname}

\author{\authname}

\date{\today}

%% Comments

\usepackage{color}

%\newif\ifcomments\commentstrue %displays comments
\newif\ifcomments\commentsfalse %so that comments do not display

\ifcomments
\newcommand{\authornote}[3]{\textcolor{#1}{[#3 ---#2]}}
\newcommand{\todo}[1]{\textcolor{red}{[TODO: #1]}}
\else
\newcommand{\authornote}[3]{}
\newcommand{\todo}[1]{}
\fi

\newcommand{\wss}[1]{\authornote{blue}{SS}{#1}} 
\newcommand{\plt}[1]{\authornote{magenta}{TPLT}{#1}} %For explanation of the template
\newcommand{\an}[1]{\authornote{cyan}{Author}{#1}}

%% Common Parts

\newcommand{\progname}{Mechatronics Engineering} % PUT YOUR PROGRAM NAME HERE
\newcommand{\authname}{Team 25, Preliminary
\\ Ahmed Nazir, nazira1
\\ Stephen Oh, ohs9
\\ Muhanad Sada, sadam
\\ Tioluwalayomi Babayeju, babayejt} % AUTHOR NAMES                  

\usepackage{hyperref}
    \hypersetup{colorlinks=true, linkcolor=blue, citecolor=blue, filecolor=blue,
                urlcolor=blue, unicode=false}
    \urlstyle{same}
                                


\begin{document}

\maketitle
\newpage

\begin{table}[hp]
\caption{Revision History} \label{TblRevisionHistory}
\begin{tabularx}{\textwidth}{llX}
\toprule
\textbf{Date} & \textbf{Developer(s)} & \textbf{Change}\\
\midrule
09/24/22 & Ahmed Nazir & Goals, Stretch Goals\\
09/25/22 & Stephen Oh & Problem Statement, Input/Output, Stakeholders\\
04/03/23 & Stephen, Ahmed & Final Revision\\
\bottomrule
\end{tabularx}
\end{table}

\newpage
\tableofcontents
\listoftables


\newpage
\section{Problem Statement}

\wss{You should check your problem statement with the
\href{https://github.com/smiths/capTemplate/blob/main/docs/Checklists/ProbState-Checklist.pdf}
{problem statement checklist}.}
\wss{You can change the section headings, as long as you include the required information.}
%Stephen updated this section, Sunday afternoon
Competitive engineering extracurricular activities at the University level revolve around the design, build,
test, learn (DBTL) cycle. In teams that have not yet reached maturity in completing high quality and
timely DBTL cycles, inexperience in the testing phase causes undesirable inefficiencies in testing
workflow such as unstandardized testing styles, collection of data, and storage of data. Such
inefficiencies also affect team workflow performance at the macro scale due to a reduced ability to
clearly view aggregate trends from unstandardized testing workflows. 

\subsection{Problem}

Our capstone group recognizes the challenge teams face with the under-allocation of resources for testing, and seek to produce a solution that reduces the time to obtain and store quality testing data and extract value from the raw data they collect. \\ 


\subsection{Inputs and Outputs}

\wss{Characterize the problem in terms of ``high level'' inputs and outputs.  
Use abstraction so that you can avoid details.}
%Stephen updated this section, Sunday afternoon

The solution will collect raw data as input and output Key Performance Indicators (KPI) statistics generated from test data. \\

Raw data input to the system will normally be aggregated upon completion of the test. The system will then communicate the contents of the aggregated data to a computing device. Users will have the opportunity to provide input on the raw data collected before saving it to a database. This input should allow the user to add contextual details to the test data to describe what the test is doing. The user will then submit the data to a database with additional information to specify key details of the test. A visualization platform will then allow users to view generated KPI statistics on the test data stored in the database. \\

Inputs: \\
A. User login to link test data to team member. \\
B. Raw data collected by testing sensors. \\
C. Communication method mode selection between wired versus wireless transfer of raw test data between the hardware and the desktop application. \\
%C. User modifications of raw test data file to maintain consistency in data through data point modifications. \\
D. User notes and picture on test data file to specify contextual details of test data collected. 

\newpage
%E. Data processing mode select between generating KPI metrics in real time, live data collection, versus generating KPI metrics in post with data from a complete test. \\

Outputs: \\
A. Tabular view of most recently collected test data over the test duration time on user application upon test completion. \\
B. Test data storage in the database. \\
C. Graphical view on visualization platform of test data over time. \\
D. Tabular view on visualization platform of Key Performance Indicators. \\
E. Visual indication from electrical hardware conveying device on/off state and wireless connection state. 
%C. Historical trends of related test Key Performance Indicator results. \\
%D. Testing trends across sub-teams. \newpage

\subsection{Stakeholders}

Formula Electric teams exemplify a highly technical, extra-curricular engineering team at the University level. On a yearly basis, these teams participate in competitions organized by a governing body, Formula SAE, who judge the quality and effectiveness of engineering ideas, designs, fabrications, and tests applied on a team's vehicle. \\

Specifically, ideal candidates who can benefit from our capstone will be Formula Electric teams with 0-4 years of competition experience. Teams without four years of experience are ideal because they typically cannot finish comprehensive tests to validate sub-team designs and overall vehicle design in a competition year. \\

As a result, our primary stakeholder will be McMaster's Formula Electric team because the team has 3 years of competition experience and faces difficulty validating all aspects of the vehicle through comprehensive testing. \\




\subsection{Environment}
The editing environment for the Arduino will be the Arduino Software IDE which has a text editor for writing code, a serial monitor to receive messages, a toolbar for common functions, and multiple menus for various applications. In this environment, the Arduino can be programmed to communicate with the user interface application running in parallel. \\

Our main environment for software programming will be Visual Studio Code. This is where we will write our code to run functions that will program our Arduino and other functions that we will need. It will also be used to write SQL code which can continually receive data from our hardware components.


\wss{Hardware and software}
\newpage
\section{Goals}
    \begin{longtable}{|p{6cm}|p{10cm}|}
    \hline
    \multicolumn{1}{|c|}{\textbf{Goals}} & \multicolumn{1}{c|}{\textbf{Explanation}} 
    \\ \hline
    Electrical Hardware capable of collecting sensor test data
    & The electrical hardware should be able to obtain and record sensor test data for the main measurements desired. The hardware should be able to collect vibration, shock, temperature and humidity measurements
    \newline                                
    \\ \hline
    Wired and wireless hardware to laptop connection
    & The hardware can support user selection between connection via a wired connection between the hardware to a laptop running the application or a wireless connection between the hardware and the laptop
    \newline                                
    \\ \hline
    Wireless connection failsafe test data storage
    & Electrical Hardware can save a complete data set from the previous test in a local memory module
    \newline                                
    \\ \hline
    Electrical connection mechanism between the electrical hardware and sensor conductors
    & The electrical hardware allows the user to fasten and unfasten sensor conductors to support varying test setups
    \newline                                
    \\ \hline
    A user interface which interacts and submits the data                                 
    & A modern GUI that is intuitive and easy to use, which should be able to communicate with our hardware to receive the measurements from the sensor. The GUI should have two modes, the first showing the live data as it comes from the device. The second mode should allow for the user to preview the data from the test, and extract certain important parts before sending it into the database  
    \newline                                
    \\ \hline
    A record of the historical data should be organized                                
    & To keep testing consistent and comparable, all previous test data should be organized into a database which cannot be edited to prevent users from tampering with test results. Historical data will allow Formula teams to be able to see how certain modifications to their design can improve or hurt motor test results. 
    \newline                                
    \\ \hline
    The final product should have a website which displays the data in a useful manner                                
    & Once all the  tests are sent to the database, we will have a website which shows all the major test information in a data analytics platform to quickly and easily see if the vehicle is hitting targets  
    \newline                              
    \\ \hline
    Setup of our device should be quick and easy                               
    & When a user would like to conduct a test our hardware should be hassle free and not add too much time to the overall setup
    \newline                            
    \\ \hline
    \caption{Goals}
    \end{longtable}
  
    
\section{Stretch Goals}
    \begin{longtable}{|p{6cm}|p{10cm}|}
        
        \hline
        \multicolumn{1}{|c|}{\textbf{Goals}} & \multicolumn{1}{c|}{\textbf{Explanation}} 
        \\ \hline
        Custom Website
        &  Creating a GUI from scratch would make the website more customizable to our particular application in terms of UI/UX   
        \newline                              
        \\ \hline
        Critical Alert                              
        & The critical alert feature would display to the user on both the App GUI and on the screen that the current test being performed is exceeding the operating conditions of the device being tested. This will bring attention to the user and they will be able to stop the test
        \newline
        \\ \hline
        Predictive Data                              
        & Since all the testing data and test conditions are being stored in our database, we could create a machine learning model which will be able to tell us how the motor will perform under certain conditions                         
        \newline       
        \\ \hline
        Security                              
        & When data is being transmitted from our hardware device to our computer, the data should be encrypted and secure to prevent tampering with data
        \newline                           
        \\ 
        \hline
        \caption{Stretch Goals}
    \end{longtable}


\end{document}