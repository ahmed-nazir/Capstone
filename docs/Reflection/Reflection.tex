\documentclass{article}

\usepackage{tabularx}
\usepackage{booktabs}

\title{Reflection Report on \progname}

\author{\authname}

\date{}

\input{../Comments}
%% Common Parts

\newcommand{\progname}{MECHTRON 4TB6} % PUT YOUR PROGRAM NAME HERE
\newcommand{\authname}{Team 25, Formulate
\\ Ahmed Nazir, nazira1
\\ Stephen Oh, ohs9
\\ Muhanad Sada, sadam
\\ Tioluwalayomi Babayeju, babayejt} % AUTHOR NAMES                  

\usepackage{hyperref}
    \hypersetup{colorlinks=true, linkcolor=blue, citecolor=blue, filecolor=blue,
                urlcolor=blue, unicode=false}
    \urlstyle{same}
                                


\begin{document}

\hbadness=99999  % or any number >=10000
% Stephen added the above line. It increases the limit before Latex generates a warning when using the \\ command to newline. Now most paragraphs that end with a \\ command shouldn't have blue wavy lines underneath the text

\maketitle

\plt{Reflection is an important component of getting the full benefits from a
learning experience.  Besides the intrinsic benefits of reflection, this
document will be used to help the TAs grade how well your team responded to
feedback.  In addition, several CEAB (Canadian Engineering Accreditation Board)
Learning Outcomes (LOs) will be assessed based on your reflections.}

\section{Changes in Response to Feedback}

\plt{Summarize the changes made over the course of the project in response to
feedback from TAs, the instructor, teammates, other teams, the project
supervisor (if present), and from user testers.}

\plt{For those teams with an external supervisor, please highlight how the feedback 
from the supervisor shaped your project.  In particular, you should highlight the 
supervisor's response to your Rev 0 demonstration to them.}

\subsection{SRS and Hazard Analysis}
%From Smith's template

\subsection{Design and Design Documentation}
%From Smith's template

\subsection{VnV Plan and Report}
%From Smith's template

\subsection{Proof of Concept Demonstration}
%Something Stephen added
At this point of the project, the user interface application allowed the user to connect to the device, start and stop a test, and submit the collected test data to the database. \\

Upon completion of the demonstration, our team received valuable feedback on how we can increase utility in stored test data from the instructor and the TA after our POC demonstration. By changing the user interface to include test description fields for the user to convey the purpose of the test session and the ability to link a picture of the test setup used to generate data, the value of the data collected increased because users have additional information to contextualize the test data collected in the past. \\
%INSERT PICTURE if available


\subsection{Revision 0 Demonstration and Testing Session with McMaster Formula Electric}
%Something Stephen added
After implementing the feedback from the POC demonstration, our team prepared a Revision 0 iteration of the project after spending time in design, manufacturing, and testing. The Revision 0 iteration met all required technical elements of the envisioned solution. \\

Upon completion of the demonstration, our team received feedback from the instructor and the TA on how we can improve the usability of the project by optimizing elements of the user experience. In scenario's where the user must adjust the Arduino code to match a new test setup with new sensors, our team changed the interface such that the user interacts with a guided test setup interface that abstracts Arduino code generation. Whereas in the previous implementation which was challenging for the user, they were expected to directly interact with Arduino code we developed to adjust lines of code to reflect the new test setup. \\

\subsection{Testing Sessions with McMaster Formula Electric}
%Something Stephen added
Soon after the Revision 0 demonstration, our team worked with the cooling system members of the McMaster Formula Electric team to gain valuable insight on additional areas of improvement. Notably at this point of the project, the implementation was identical to the Revision 0 demonstration. \\

Upon completion of the cooling system testing sessions with the Formula Electric team, we received feedback on how the project's portability can be improved. A main concern our team noted from a member of the cooling system was the relatively large form factor of the Revision 0 implementation. It was noted that the member found the device difficult to integrate the device into the benchtop testing setup due to its size. As a result, our team reduced the area of the PCB and the 3D printed enclosure to accommodate the need for a device with a smaller form factor. \\


\subsection{Revision 1 Demonstration}
%Something Stephen added
After implementing the feedback from the Revision 0 demonstration and testing session with McMaster Formula Electric, our team prepared a Revision 1 iteration of the project after spending time in design, manufacturing, and testing. The Revision 1 iteration represents a finalized version of the project that met all required technical elements of the envisioned solution and had additional features to support user experience optimization. \\

Upon completion of the demonstration, our team received final comments on how we can educate McMaster Formula Electric members effectively and efficiently about our product from the instructor and the TA. By creating an educational training video on what typical device and user interface operation looks like, members of the McMaster Formula Electric team can quickly integrate our project into their workflows. \\

\section{Design Iteration (LO11)}

\plt{Explain how you arrived at your final design and implementation.  How did
the design evolve from the first version to the final version?} 

\section{Design Decisions (LO12)}

\plt{Reflect and justify your design decisions.  How did limitations,
 assumptions, and constraints influence your decisions?}

\section{Economic Considerations (LO23)}

\plt{Is there a market for your product? What would be involved in marketing your 
product? What is your estimate of the cost to produce a version that you could 
sell?  What would you charge for your product?  How many units would you have to 
sell to make money? If your product isn't something that would be sold, like an 
open source project, how would you go about attracting users?  How many potential 
users currently exist?}

\section{Reflection on Project Management (LO24)}

\plt{This question focuses on processes and tools used for project management.}

\subsection{How Does Your Project Management Compare to Your Development Plan}

\plt{Did you follow your Development plan, with respect to the team meeting plan, 
team communication plan, team member roles and workflow plan.  Did you use the 
technology you planned on using?}

\subsection{What Went Well?}

\plt{What went well for your project management in terms of processes and 
technology?}

\subsection{What Went Wrong?}

\plt{What went wrong in terms of processes and technology?}

\subsection{What Would you Do Differently Next Time?}

\plt{What will you do differently for your next project?}

\end{document}