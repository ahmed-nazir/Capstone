\documentclass{article}

\usepackage{booktabs}
\usepackage{tabularx}

\title{Development Test\\\progname}

\author{\authname}

\date{}

%% Comments

\usepackage{color}

%\newif\ifcomments\commentstrue %displays comments
\newif\ifcomments\commentsfalse %so that comments do not display

\ifcomments
\newcommand{\authornote}[3]{\textcolor{#1}{[#3 ---#2]}}
\newcommand{\todo}[1]{\textcolor{red}{[TODO: #1]}}
\else
\newcommand{\authornote}[3]{}
\newcommand{\todo}[1]{}
\fi

\newcommand{\wss}[1]{\authornote{blue}{SS}{#1}} 
\newcommand{\plt}[1]{\authornote{magenta}{TPLT}{#1}} %For explanation of the template
\newcommand{\an}[1]{\authornote{cyan}{Author}{#1}}

%% Common Parts

\newcommand{\progname}{Mechatronics Engineering} % PUT YOUR PROGRAM NAME HERE
\newcommand{\authname}{Team 25, Preliminary
\\ Ahmed Nazir, nazira1
\\ Stephen Oh, ohs9
\\ Muhanad Sada, sadam
\\ Tioluwalayomi Babayeju, babayejt} % AUTHOR NAMES                  

\usepackage{hyperref}
    \hypersetup{colorlinks=true, linkcolor=blue, citecolor=blue, filecolor=blue,
                urlcolor=blue, unicode=false}
    \urlstyle{same}
                                


\begin{document}

\begin{table}[hp]
\caption{Revision History} \label{TblRevisionHistory}
\begin{tabularx}{\textwidth}{llX}
\toprule
\textbf{Date} & \textbf{Developer(s)} & \textbf{Change}\\
\midrule
Sept 25 & Muhanad Sada & Workflow Plan, POC Demo Plan\\
Date2 & Name(s) & Description of changes\\
... & ... & ...\\
\bottomrule
\end{tabularx}
\end{table}

\newpage

\maketitle

\wss{Put your introductory blurb here.}

\section{Team Meeting Plan}

\section{Team Communication Plan}

\section{Team Member Roles}

\section{Workflow Plan}
Team members will use the GitHub repository dedicated for the capstone project. The feature
branch workflow will be used whenever there are any code changes except for simple fixes such as syntax errors, comments, variable renaming, etc. 
Branches will also be utilized for any significant documentation changes such as section additions/modifications and diagram insertions. 
Pull requests will be used in conjunction with branches to review/verify code and document changes.
Branches will follow the following naming structure: scope/(description), ex. feat/adding new function

\subsection{Issues}
The issues feature in GitHub is used to track all of the tasks for the project. Once the team or individual members 
identify a task, an issue will be created. When creating an issue, a team member will select one of the issue templates 
based on the scope of the task. There are a total of five templates:

\begin{itemize}
	\item Bug report - any tasks used to report a bug and fix it
	\item Feature request – any tasks that involve requesting and implementing a feature
	\item Enhancement - any tasks that require updating code for enhancement purposes
	\item Documentation – tasks that involve adding or editing documentation
	\item Miscellaneous – any tasks that are not covered under the scope of the other templates
\end{itemize}

\subsection{Project Board}
The project board is used to organize and identify the status of each task. 
The project contains five columns each describing the current status of the issue:

\begin{itemize}
	\item To-do - When tasks are first created, they are placed in this list
	\item In-progress – The issue has been assigned to a team member and is currently being worked on
	\item In-Review – The work has been completed and now to needs to be reviewed
	\item Done – Once team member(s) review and approve the changes, the issue will be moved to this stage
	\item Outdated/Ignored – issues that were created but later determined to be unnecessary 
\end{itemize}

\section{Proof of Concept Demonstration Plan}

The proof of concept demonstration should prove four essential 
functionalities of the product. The first is the ability of sensors to measure 
desired data and send that information to the hardware. The second is having the capability to receive/send 
data at three different levels, which includes hardware, desktop application, and database. 
The POC should be able to show that hardware can receive information from a sensor and send that 
information to a simple desktop application. The application should then be able to receive that 
data and display it on the GUI. At this point, the application would be able to send that 
information to a database, which is populated accordingly. The third ability, 
is to show live data on the application’s GUI, however implementation difficulties are expected. 
This is due to the tediousness of creating a connection that provides both smooth and continuous 
data transfer between the hardware and the application. In addition to these functionalities, there 
is the risk of being constrained in testing as we might not have access to a formula E car or it will 
be difficult to duplicate. Therefore, the POC should also have a testing environment that mirrors the 
conditions/setup of the mechanical parts of a Formula E car upon taking measurements. If the implementation of 
the above essential abilities and testing environment are verified then the level of confidence of creating 
a successful data automation product. 

\section{Technology}

\begin{itemize}
\item Specific programming language
\item Specific linter tool (if appropriate)
\item Specific unit testing framework
\item Investigation of code coverage measuring tools
\item Specific plans for Continuous Integration (CI), or an explanation that CI
  is not being done
\item Specific performance measuring tools (like Valgrind), if
  appropriate
\item Libraries you will likely be using?
\item Tools you will likely be using?
\end{itemize}

\section{Coding Standard}

\section{Project Scheduling}

\wss{How will the project be scheduled?}

\end{document}