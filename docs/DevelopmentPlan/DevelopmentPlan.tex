\documentclass[12pt]{article}

\usepackage{booktabs}
\usepackage{tabularx}
\usepackage{float}
\usepackage{fullpage}
\usepackage{longtable}



\date{}

%% Comments

\usepackage{color}

%\newif\ifcomments\commentstrue %displays comments
\newif\ifcomments\commentsfalse %so that comments do not display

\ifcomments
\newcommand{\authornote}[3]{\textcolor{#1}{[#3 ---#2]}}
\newcommand{\todo}[1]{\textcolor{red}{[TODO: #1]}}
\else
\newcommand{\authornote}[3]{}
\newcommand{\todo}[1]{}
\fi

\newcommand{\wss}[1]{\authornote{blue}{SS}{#1}} 
\newcommand{\plt}[1]{\authornote{magenta}{TPLT}{#1}} %For explanation of the template
\newcommand{\an}[1]{\authornote{cyan}{Author}{#1}}

%% Common Parts

\newcommand{\progname}{Mechatronics Engineering} % PUT YOUR PROGRAM NAME HERE
\newcommand{\authname}{Team 25, Preliminary
\\ Ahmed Nazir, nazira1
\\ Stephen Oh, ohs9
\\ Muhanad Sada, sadam
\\ Tioluwalayomi Babayeju, babayejt} % AUTHOR NAMES                  

\usepackage{hyperref}
    \hypersetup{colorlinks=true, linkcolor=blue, citecolor=blue, filecolor=blue,
                urlcolor=blue, unicode=false}
    \urlstyle{same}
                                


\begin{document}
\title{Development Plan\\\progname}

\author{\authname}
\date{\today}

\maketitle
\newpage

\begin{table}[hp]
\caption{Revision History} \label{TblRevisionHistory}
\begin{tabularx}{\textwidth}{llX}
\toprule
\textbf{Date} & \textbf{Developer(s)} & \textbf{Change}\\
\midrule
09/25/22 & Muhanad Sada & Workflow Plan, POC Demo Plan, Technology\\
09/25/22 & Ahmed Nazir & Technology\\
04/03/23 & Ahmed Nazir & Final Revision\\
04/04/23 & Muhanad Sada & Updated team roles, programming libraries, and software tools \\
\bottomrule
\end{tabularx}
\end{table}

\newpage
\tableofcontents
\listoftables

\newpage



\wss{Put your introductory blurb here.}

\section{Team Meeting Plan}
%Stephen updated this section, Sunday afternoon
The team plans on having in person meetings weekly on Mondays from 2:30PM - 4:30 PM. These meetings will be used to catch up on work done in the last week, next steps and any important updates. If in person meetings are not possible, we will conduct them through Microsoft Teams. This weekly meeting is mandatory but we may also have other meetings throughout the week depending on specific project needs. All meetings will have an agenda and each team lead will give updates on the next steps. 

\section{Team Communication Plan}
Our team will use Microsoft Teams as our main method of communication via chat and GitHub's issue tracker to delegate specific deliverable tasks.

\section{Team Member Roles}
%Stephen updated this section, Sunday afternoon
The following roles were selected based on the areas of expertise and previous experience of each team member. 
For example, Muhanad had the most experience with software related projects and it was determined that his main focus will be the user interface accordingly.
\begin{itemize}
	\item Stephen Oh - Electrical and Mechanical Hardware Lead
	\item Ahmed Nazir - Mechanical Hardware Lead, Firmware Lead
	\item Muhanad Sada - Software User Interface Lead, Database Lead
	\item Tioluwalayomi Babayeju - Software Dashboard Platform lead
\end{itemize}


\section{Workflow Plan}
%Stephen updated this section, Sunday afternoon
Our team will use the GitHub repository dedicated for the capstone project. The feature
branch workflow will be used whenever there are any code changes except for simple fixes such as syntax errors, comments, variable renaming, etc. 
Branches will be utilized for significant documentation changes such as section additions or modifications and diagram insertions. 
Pull requests will be used in conjunction with branches to review code and document changes.
Branches will follow the following the naming structure of scope/description. For example: Feat/Adding new function.

\newpage

\subsection{Issues}
The issues feature in GitHub is used to track all project tasks. Once the team or individual members identify a task, an issue will be created. In order to keep issues updated with the corresponding commit, the following commit message structure will be followed: (scope): [\#issue number] description. When creating an issue, a team member will select one of the issue templates based on the scope of the task. There are a total of five templates: 

\begin{itemize}
	\item Bug report - any tasks used to report a bug and fix it
	\item Feature – any tasks that involve requesting and implementing a feature
	\item Enhancement - any tasks that require updating code for enhancement purposes
	\item Documentation – tasks that involve adding or editing documentation
	\item Miscellaneous – any tasks that are not covered under the scope of the other templates
\end{itemize}

\noindent
%Stephen updated this section, Sunday afternoon
Labels will be added to issues to identify the affected product subsystem. The following labels will be utilized: 
database, desktop application, GUI, hardware, and mechanical design.

\subsection{Project Board}
The project board will be used to organize and identify the status of each task. 
The project contains five columns each describing the current status of the issue:

\begin{itemize}
	\item To-do - When tasks are first created, they are placed in this list
	\item In-progress – The issue has been assigned to a team member and is currently being worked on
	\item In-Review – The work has been completed and now to needs to be reviewed
	\item Done – Once team member(s) review and approve the changes, the issue will be moved to this stage
	\item Outdated/Ignored – issues that were created but later determined to be unnecessary 
\end{itemize}

\newpage

\section{Proof of Concept Demonstration Plan}
%Stephen updated this section, Sunday afternoon

The proof of concept demonstration will prove three essential product
functionalities. The first is the sensor's ability to measure and output the  
desired data and to the computing hardware. The second capability will be to receive and send 
data between three different subsystems, which includes the electrical hardware, desktop application, and database. 
The POC will show that hardware can receive information from a sensor and send the collected sensor data 
to a simple desktop application. The application should then be able to receive and display the sensor 
data on the GUI. At this point, the application gives the user the ability to send the sensor data
information to a database, populating relevant tables accordingly. 
%The third ability, %% We may not need this section on implementation difficulties.
%is to show live data on the application’s GUI, however implementation difficulties are expected. 
%This is due to the tediousness of creating a connection that provides both smooth and continuous 
%data transfer between the hardware and the application. In addition to these functionalities, there 
%is the risk of being constrained in testing as we might not have access to a Formula E car or it will 
%be difficult to duplicate. Therefore, the POC should also have a testing environment that mirrors the 
%conditions/setup of the mechanical parts of a Formula E car upon taking measurements. If the implementation of 
%the above essential abilities and testing environment are verified then the level of confidence of creating 
%a successful data automation product will be achieved. 

\section{Technology}

\subsection{Programming Language}

The programming language of choice is Python. Python provides the ideal balance between data analysis, data manipulation capabilities, libraries available for graphical user interfaces, and documentation widely available to support development.  

\subsubsection{Libraries}
%Stephen updated this section, Sunday afternoon
\begin{itemize}
	\item Pandas: Data manipulation and formatting library
	%\item Matplotlib: Math library to help visualize data in graphs
	\item PyQT: Python GUI library
	\item Serial: Serial communication library
	\item Socket: Networking library to connect via wireless
	\item Pyodbc: Library for connecting and accessing ODBC databases
	%\item PyFirmata: Python library that allows serial communication between Python and the Arduino board
\end{itemize}

\newpage
\subsection{Software Tools}
%Stephen updated this section, Sunday afternoon
%Tio updated this section, Monday afternoon
To assist with developing our product we will be using the following software tools.


	\begin{longtable}{|p{6cm}|p{10cm}|}
		\hline
		\multicolumn{1}{|c|}{\textbf{Tool Name}} & \multicolumn{1}{c|}{\textbf{Explanation}} 
		\\ \hline
		Autodesk Inventor
		&  Autodesk's Inventor will be used to complete CAD component and assembly design work to create our hardware  
		\newline                              
		\\ \hline
		KiCad
		& Kicad's schematic editor will be used to complete the electrical design of the electrical hardware component's power and signal connections. Kicad will also be used to generate the PCB layouts and trace connections
		\newline                              
		\\ \hline
		VSCode
		&  Our team will be using VSCode as our primary code editor because of large number of extensions available. VSCode also integrates well with Git and GitHub
		\newline                              
		\\ \hline
		GitHub Desktop
		&  GitHub desktop is an easy GUI to use and interacts with our GitHub repo to make code and file editing more efficient
		\newline                              
		\\ \hline
		PyQT Designer
		&  PyQT Designer will be used by our team to design the GUI in a fast and effective way
		\newline                     
		\\ \hline
		%Pylint
		%&  The linter extension we plan on using is Pylint, it will keep our coding style consistent between different people and it is specifically made for Python
		%\newline                              
		Arduino IDE
		&  Software used to write and upload firmware code onto the Arduino board
		\newline                     
		\\ \hline
		Microsoft Azure
		&  Cloud computing service that allows you to create, connect to, and manage database/storage instances
		\newline                     
		\\ \hline
		Microsoft Power Bi
		&  Microsoft Power Bi is an interactive data visualization software product which we will be using to create dashboards to help the user visualize their test data.
		\newline                              
		\\ \hline
		\caption{Software Tools}
	\end{longtable}


%Ahmed updated this section Monday Morning
\newpage

\subsection{Hardware}
\begin{itemize}
	\item Arduino UNO: Open-source microcontroller 
	\item ESP8266 (NodeMCU 1.0): Wi-Fi module which gives Arduino Wi-Fi access
	\item GeekStory MicroSD Adapter: MicroSD adapter which gives Arduino connection to a local memory storage
	\item SanDisk MicroSD: 32GB Arduino local memory storage 
	\item Sensors: Measuring desired values
	\item 3D Printer: Printing CAD designed electronics enclosure and supporting mechanical components
\end{itemize}



\section{Coding Standard}


	\begin{longtable}{|p{6cm}|p{10cm}|}
		\hline
		\multicolumn{1}{|c|}{\textbf{Coding Standard}} & \multicolumn{1}{c|}{\textbf{Explanation}} 
		\\ \hline
		Coding readability
		&  
		\begin{itemize}
			\item Capitalize SQL key words to differentiate them from columns and table names
			\item Avoiding deep nested loops to help make it easier to follow and read
			\item Avoid creating long function that do multiple tasks, instead make small functions which do a single task.
			\item Writing comments consistently explaining what is happening in each section of the code
			\item Using meaningful variables to help make code more understandable
			\item Using appropriate naming conventions
		\end{itemize}                                 
		\\ \hline
		Module headers
		&  
		\begin{itemize}
			\item Creating module names
			\item Creation date
			\item History of changes to modules
			\item Summary of what each module does
			\item Functions and variable name changed in each modules
			\item Tracking various issues in each module
		\end{itemize} 
								
		\\ \hline
		Proper indentation
		&
		\begin{itemize}
			\item Proper space between two function arguments after a comma
			\item Proper indentation and spaces for nested blocks in code
			\item All braces should start from a new line and the end of the braces also start on a new line
		\end{itemize}
									
		\\ \hline
	\end{longtable}


\section{Project Scheduling}
%Stephen updated this section, Sunday afternoon

\wss{How will the project be scheduled?}

Weekly meetings to identify product development will be used to manage our project's development progress and meet key development deadlines. We will go over each person's delegated tasks and see if they were met during the meeting. If the target was not met, we analyze how the setback affects the development schedule and allocate additional resources accordingly to ensure completion. We will have major deadlines added to a shared calendar on teams to maintain high deadline visibility as we work on our project. At the beginning of each month during our Monday meeting, we will discuss potential roadblocks and what must be completed for the month. Delegating tasks will be made during weekly meetings and are based on the schedules of each individual and their expertise.



\end{document}