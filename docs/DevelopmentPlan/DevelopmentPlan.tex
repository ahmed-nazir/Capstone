\documentclass{article}

\usepackage{booktabs}
\usepackage{tabularx}
\usepackage{float}

\title{Development Test\\\progname}

\author{\authname}

\date{}

%% Comments

\usepackage{color}

%\newif\ifcomments\commentstrue %displays comments
\newif\ifcomments\commentsfalse %so that comments do not display

\ifcomments
\newcommand{\authornote}[3]{\textcolor{#1}{[#3 ---#2]}}
\newcommand{\todo}[1]{\textcolor{red}{[TODO: #1]}}
\else
\newcommand{\authornote}[3]{}
\newcommand{\todo}[1]{}
\fi

\newcommand{\wss}[1]{\authornote{blue}{SS}{#1}} 
\newcommand{\plt}[1]{\authornote{magenta}{TPLT}{#1}} %For explanation of the template
\newcommand{\an}[1]{\authornote{cyan}{Author}{#1}}

%% Common Parts

\newcommand{\progname}{Mechatronics Engineering} % PUT YOUR PROGRAM NAME HERE
\newcommand{\authname}{Team 25, Preliminary
\\ Ahmed Nazir, nazira1
\\ Stephen Oh, ohs9
\\ Muhanad Sada, sadam
\\ Tioluwalayomi Babayeju, babayejt} % AUTHOR NAMES                  

\usepackage{hyperref}
    \hypersetup{colorlinks=true, linkcolor=blue, citecolor=blue, filecolor=blue,
                urlcolor=blue, unicode=false}
    \urlstyle{same}
                                


\begin{document}

\begin{table}[hp]
\caption{Revision History} \label{TblRevisionHistory}
\begin{tabularx}{\textwidth}{llX}
\toprule
\textbf{Date} & \textbf{Developer(s)} & \textbf{Change}\\
\midrule
Sept 25 & Muhanad Sada & Workflow Plan, POC Demo Plan\\
Date2 & Name(s) & Description of changes\\
... & ... & ...\\
\bottomrule
\end{tabularx}
\end{table}

\newpage

\maketitle

\wss{Put your introductory blurb here.}

\section{Team Meeting Plan}
Our team plans on having in person meetings weekly on Mondays from 2:30-4:30PM. These meetings will be used to catch up on work done in the last week, next steps and any important updates. If in person meetings are not possible we will conduct them through Microsoft Teams. This weekly meeting is mandatory but we might also have other meetings throughout the week depending on what stage we are in for the project. All meetings will contain an agenda and each team lead will give updates on next steps 

\section{Team Communication Plan}
Our team will be using a Microsoft Teams group as our main method of chat, and to delegate tasks we will be utilizing GitHubs project tracker

\section{Team Member Roles}
\begin{itemize}
	\item Stephen Oh - Team Liason, Software/Hardware Team
	\item Ahmed Nazir - Software/Hardware Team, Inventor Expert
	\item Muhanad Sada - Software/Hardware Team, Git Expert
	\item Tioluwalayomi Babayeju - Software/Hardware Team
\end{itemize}


\section{Workflow Plan}
Team members will use the GitHub repository dedicated for the capstone project. The feature
branch workflow will be used whenever there are any code changes except for simple fixes such as syntax errors, comments, variable renaming, etc. 
Branches will also be utilized for any significant documentation changes such as section additions/modifications and diagram insertions. 
Pull requests will be used in conjunction with branches to review/verify code and document changes.
Branches will follow the following naming structure: scope/(description), ex. feat/adding new function

\subsection{Issues}
The issues feature in GitHub is used to track all of the tasks for the project. Once the team or individual members 
identify a task, an issue will be created. When creating an issue, a team member will select one of the issue templates 
based on the scope of the task. There are a total of five templates:

\begin{itemize}
	\item Bug report - any tasks used to report a bug and fix it
	\item Feature request – any tasks that involve requesting and implementing a feature
	\item Enhancement - any tasks that require updating code for enhancement purposes
	\item Documentation – tasks that involve adding or editing documentation
	\item Miscellaneous – any tasks that are not covered under the scope of the other templates
\end{itemize}

\subsection{Project Board}
The project board is used to organize and identify the status of each task. 
The project contains five columns each describing the current status of the issue:

\begin{itemize}
	\item To-do - When tasks are first created, they are placed in this list
	\item In-progress – The issue has been assigned to a team member and is currently being worked on
	\item In-Review – The work has been completed and now to needs to be reviewed
	\item Done – Once team member(s) review and approve the changes, the issue will be moved to this stage
	\item Outdated/Ignored – issues that were created but later determined to be unnecessary 
\end{itemize}

\section{Proof of Concept Demonstration Plan}

The proof of concept demonstration should prove four essential 
functionalities of the product. The first is the ability of sensors to measure 
desired data and send that information to the hardware. The second is having the capability to receive/send 
data at three different levels, which includes hardware, desktop application, and database. 
The POC should be able to show that hardware can receive information from a sensor and send that 
information to a simple desktop application. The application should then be able to receive that 
data and display it on the GUI. At this point, the application would be able to send that 
information to a database, which is populated accordingly. The third ability, 
is to show live data on the application’s GUI, however implementation difficulties are expected. 
This is due to the tediousness of creating a connection that provides both smooth and continuous 
data transfer between the hardware and the application. In addition to these functionalities, there 
is the risk of being constrained in testing as we might not have access to a formula E car or it will 
be difficult to duplicate. Therefore, the POC should also have a testing environment that mirrors the 
conditions/setup of the mechanical parts of a Formula E car upon taking measurements. If the implementation of 
the above essential abilities and testing environment are verified then the level of confidence of creating 
a successful data automation product. 

\section{Technology}

\subsection{Programming Language}

The programming language of choice is Python. Python is the perfect language for data analysis and data manipulation. Python contains many libraries specifically for data analysis. 

\subsubsection{Libraries}
\begin{itemize}
	\item Pandas: Data manipulation library
	\item Matplotlib: Math library to help visualize data in graphs
	\item PyQT: Python GUI library
\end{itemize}


\subsection{Tools}
To assist with developing our product we will be using the following tools

\begin{table}[!hbt]
	\centering
	\begin{tabular}{|p{4cm}|p{8cm}|}
	\hline
	\multicolumn{1}{|c|}{\textbf{Tool Name}} & \multicolumn{1}{c|}{\textbf{Explanation}} 
	\\ \hline
	Autodesk Inventor
	&  Inventor will be used for all the CAD design work to create our hardware  
	\newline                              
	\\ \hline
	VSCode
	&  Our team will be using VSCode as our primary code editor because of the vast extensions it has. VSCode also integrates nicely with git and GitHub
	\newline                              
	\\ \hline
	GitHub Desktop
	&  GitHub desktop is an easy to use GUI which interacts with our github repo and makes editing code and files faster
	\newline                              
	\\ \hline
	Pylint
	&  The linter extension we plan on using is Pylint, it will keep our coding style consistent between different people and it is specifically made for Python
	\newline                              
	\\ \hline
	\end{tabular}
\end{table}




\section{Coding Standard}

\begin{table}[H]
	\centering
	\begin{tabular}{|p{4cm}|p{8cm}|}
	\hline
	\multicolumn{1}{|c|}{\textbf{Coding Standard}} & \multicolumn{1}{c|}{\textbf{Explanation}} 
	\\ \hline
	Coding readability
	&  
	\begin{itemize}
		\item Capitalize SQL special words to differentiate them from columns and table names
		\item Avoiding deep nested loops to help make it easier to follow and read
		\item Not using lengthy functions, so avoiding the use of functions doing multiple tasks
		\item Writing comments consistently explaining what is happening in each section of the code
		\item Using meaningful variables to help make code more understandable
		\item Using appropriate naming conventions
	\end{itemize}                                 
	\\ \hline
	Module headers
	&  
	\begin{itemize}
		\item Creating module name
		\item Creation date
		\item History of changes to modules
		\item Summary of what each module does
		\item Functions and variables changed in each modules
	\end{itemize} 
                             
	\\ \hline
	Proper indentation
	&
	\begin{itemize}
		\item Proper space between two function arguments after a comma
		\item Proper indentation and spaces for nested blocks in code
		\item All braces should start from a new line and the end of the braces also start on a new line
	\end{itemize}
	                             
	\\ \hline
	\end{tabular}
\end{table}

\section{Project Scheduling}

\wss{How will the project be scheduled?}

\end{document}