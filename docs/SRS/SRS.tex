\documentclass[12pt]{article}

\usepackage{amsmath, mathtools}
\usepackage{amsfonts}
\usepackage{amssymb}
\usepackage{graphicx}
\usepackage{colortbl}
\usepackage{xr}
\usepackage{hyperref}
\usepackage{longtable}
\usepackage{xfrac}
\usepackage{tabularx}
\usepackage{float}
\usepackage{siunitx}
\usepackage{booktabs}
\usepackage{caption}
\usepackage{pdflscape}
\usepackage{afterpage}

\usepackage[round]{natbib}

%\usepackage{refcheck}

\hypersetup{
    bookmarks=true,         % show bookmarks bar?
    colorlinks=true,       % false: boxed links; true: colored links
    linkcolor=red,          % color of internal links (change box color with linkbordercolor)
    citecolor=green,        % color of links to bibliography
    filecolor=magenta,      % color of file links
    urlcolor=cyan           % color of external links
}

\input{../Comments}
%% Common Parts

\newcommand{\progname}{MECHTRON 4TB6} % PUT YOUR PROGRAM NAME HERE
\newcommand{\authname}{Team 25, Formulate
\\ Ahmed Nazir, nazira1
\\ Stephen Oh, ohs9
\\ Muhanad Sada, sadam
\\ Tioluwalayomi Babayeju, babayejt} % AUTHOR NAMES                  

\usepackage{hyperref}
    \hypersetup{colorlinks=true, linkcolor=blue, citecolor=blue, filecolor=blue,
                urlcolor=blue, unicode=false}
    \urlstyle{same}
                                


% For easy change of table widths
\newcommand{\colZwidth}{1.0\textwidth}
\newcommand{\colAwidth}{0.13\textwidth}
\newcommand{\colBwidth}{0.82\textwidth}
\newcommand{\colCwidth}{0.1\textwidth}
\newcommand{\colDwidth}{0.05\textwidth}
\newcommand{\colEwidth}{0.8\textwidth}
\newcommand{\colFwidth}{0.17\textwidth}
\newcommand{\colGwidth}{0.5\textwidth}
\newcommand{\colHwidth}{0.28\textwidth}

% Used so that cross-references have a meaningful prefix
\newcounter{defnum} %Definition Number
\newcommand{\dthedefnum}{GD\thedefnum}
\newcommand{\dref}[1]{GD\ref{#1}}
\newcounter{datadefnum} %Datadefinition Number
\newcommand{\ddthedatadefnum}{DD\thedatadefnum}
\newcommand{\ddref}[1]{DD\ref{#1}}
\newcounter{theorynum} %Theory Number
\newcommand{\tthetheorynum}{T\thetheorynum}
\newcommand{\tref}[1]{T\ref{#1}}
\newcounter{tablenum} %Table Number
\newcommand{\tbthetablenum}{T\thetablenum}
\newcommand{\tbref}[1]{TB\ref{#1}}
\newcounter{assumpnum} %Assumption Number
\newcommand{\atheassumpnum}{P\theassumpnum}
\newcommand{\aref}[1]{A\ref{#1}}
\newcounter{goalnum} %Goal Number
\newcommand{\gthegoalnum}{P\thegoalnum}
\newcommand{\gsref}[1]{GS\ref{#1}}
\newcounter{instnum} %Instance Number
\newcommand{\itheinstnum}{IM\theinstnum}
\newcommand{\iref}[1]{IM\ref{#1}}
\newcounter{reqnum} %Requirement Number
\newcommand{\rthereqnum}{P\thereqnum}
\newcommand{\rref}[1]{R\ref{#1}}
\newcounter{nfrnum} %NFR Number
\newcommand{\rthenfrnum}{NFR\thenfrnum}
\newcommand{\nfrref}[1]{NFR\ref{#1}}
\newcounter{lcnum} %Likely change number
\newcommand{\lthelcnum}{LC\thelcnum}
\newcommand{\lcref}[1]{LC\ref{#1}}

\usepackage{fullpage}


\begin{document}

\title{Software Requirements Specification \progname: Formulate} 

\author{\authname}
\date{\today}
	
\maketitle

~\newpage

\pagenumbering{roman}

\tableofcontents

~\newpage

\section*{Revision History}

\begin{tabularx}{\textwidth}{p{3cm}p{2cm}X}
\toprule {\bf Date} & {\bf Version} & {\bf Notes}\\
\midrule
Date 1 & 1.0 & Notes\\
Date 2 & 1.1 & Notes\\
\bottomrule
\end{tabularx}

~\newpage

\section{Introduction}

\subsection{Document Purpose}

This document provides the set of Software Requirements Specifications (SRS) used to describe the system developed to assist testing efforts in technical teams. Both hardware and software system requirements were included to fully specify all system requirements. \\

The user can expect to understand the system behavior under expected use cases, the functional and non-functional requirements the system must adhere to, and a phase in development plan.\\

\subsection{Project Description}

Effective test data collection and storage is a common challenge extra-curricular teams face in the technical domain. In teams who do not invest in streamlining data collection and storage, teams cannot fully utilize test data to validate designs. As a result, teams encounter difficulty proving design validity during competition, experience reduced competitiveness when presenting an under-validated system, and fail to generate trends on aggregated test data to efficiently find areas of improvement in design. \\

Project "Formulate" enables engineering teams to streamline data collection and storage, resulting in testing overhead reduction and increased control of raw test data gathered by automating aspects of the testing procedure.\\

\subsection{Project Scope}

Project Formulate aims to provide the McMaster Formula Electric team with a well-documented and complete system. To accomplish the project goals within an 8 month timeline, the following scope of requirements were developed to set clear boundaries on deliverables.\\

1. Documentation for tool integration into testing workflows for common tests.\\
2. Hardware capable of collecting data from test equipment.\\
3. User interface to interact with raw data and submit the data to a database.\\
4. Record of organized, historical data.\\
5. Visualization of test data stored in a database with auto-generated KPI metrics.\\
6. Short setup time to integrate device into testing workflow, regardless of technical background.\\

  Out of Scope Items:\\
1. Custom website to visualize test data results stored in a database.\\
2. Security through data encryption.\\
3. Predictive intelligence to estimate if rate of test data collected is on track to produce a fully validated product.\\

\subsection{Table of Symbols}

\renewcommand{\arraystretch}{1.2}
%\noindent \begin{tabularx}{1.0\textwidth}{l l X}
\noindent \begin{longtable*}{l l p{12cm}} \toprule
\textbf{Symbol} & \textbf{Unit} & \textbf{Description}\\
\midrule 
$A_C$ & \si[per-mode=symbol] {\square\metre} & coil surface area\\
\bottomrule
\end{longtable*}


\subsection{Abbreviations and Acronyms}

\renewcommand{\arraystretch}{1.2}
%\noindent \begin{tabular}{l l} 
\noindent \begin{longtable*}{l p{13cm}} 
  \toprule		
  \textbf{Symbol} & \textbf{Description}\\
  \midrule 
  A & Assumption\\
  DD & Data Definition\\
  GD & General Definition\\
  GS & Goal Statement\\
  IM & Instance Model\\
  LC & Likely Change\\
  PS & Physical System Description\\
  R & Requirement\\
  SRS & Software Requirements Specification\\
  DBTL & Design Build Test Learning\\
  KPI & Key Performance Indicators\\
  \bottomrule
%\end{tabular}\\
\end{longtable*}



\section{User Characteristics}

\subsection{Stakeholders}


\subsection{Use Cases} 

\subsection{User Consideration}

\subsection{Impact}


\section{Requirements}

\plt{The requirements refine the goal statement.  They will make heavy use of
  references to the instance models.}

This section provides the functional requirements, the business tasks that the
software is expected to complete, and the nonfunctional requirements, the
qualities that the software is expected to exhibit.

\subsection{Functional Requirements}

\subsubsection{Hardware} 

\begin{itemize}
  \item[RH\refstepcounter{reqnum}\thereqnum:] The device should contain a rechargeable battery
  
  
  \item[RH\refstepcounter{reqnum}\thereqnum:] The device should have a screen to display the current status to the user
  
  \item[RH\refstepcounter{reqnum}\thereqnum:] The device should easily mount to the base of a Formula SAE car
  
  \item[RH\refstepcounter{reqnum}\thereqnum:] The device should connect to a PC wirelessly to transmit data
  
  \item[RH\refstepcounter{reqnum}\thereqnum:] 
  
  \end{itemize}


\subsubsection{Desktop Application}

\subsubsection{Data Analytics Platform}

\newpage
\subsection{Nonfunctional Requirements}

\noindent\begin{itemize}

\subsubsection{Usability} 

    \item[NFR\refstepcounter{nfrnum}\thenfrnum:]
    \textbf{Ease of Learning}\\
    The user will be able to learn the tool's operation quickly to integrate into their testing workflow efficiently\\

    \item[NFR\refstepcounter{nfrnum}\thenfrnum:]
    \textbf{Ease of Use}\\
    The system will be fast at processing data such that additional overhead through the use of the tool is less than if all components of the testing workflow were completed individually.\\

\subsubsection{Performance} 

    \item[NFR\refstepcounter{nfrnum}\thenfrnum:]
    \textbf{Speed}\\
    The system bandwidth will be high enough to support testing equipment with high data collection frequencies.\\

    \item[NFR\refstepcounter{nfrnum}\thenfrnum:]
    \textbf{Reliability and Availability}\\
    The system will be fail-safe to withstand single point of failures in components with high probability of operational failure.\\

\subsubsection{Operational}

    \item[NFR\refstepcounter{nfrnum}\thenfrnum:]
      \textbf{Expected Technological Environment}\\
    The tool will be able to facilitate a variety of tests using a range of equipment, as long as the equipment is compatible with the data measuring hardware.\\

    \item[NFR\refstepcounter{nfrnum}\thenfrnum:]
    \textbf{Expected Physical Environment}
    The system will be operational under a wide range of temperatures and operational vibrations.\\

\subsubsection{Maintainability and Portability}

    \item[NFR\refstepcounter{nfrnum}\thenfrnum:]
      \textbf{Maintainability}\\
    The system will be modular and have low cohesion such that users can adapt elements of the tool's hardware and software infrastructure to current needs without breaking other elements.\\

    \item[NFR\refstepcounter{nfrnum}\thenfrnum:]
    \textbf{Portability}\\
    The user's ability to conduct tests will not be affected by the physical constraints from the tool.\\
  
\subsubsection{Security}

    \item[NFR\refstepcounter{nfrnum}\thenfrnum:]
    \textbf{Software Integrity}\\
    The system will be secure against malicious spam aimed at reducing validity of aggregate test data stored in the database.\\

\subsubsection{Cultural and Political}

    N/A\\

\subsubsection{Legal}

    N/A\\zzz

\end{itemize}


\section{Likely Changes}    

\noindent \begin{itemize}

\item[LC\refstepcounter{lcnum}\thelcnum\label{LC_meaningfulLabel}:] \plt{Give
    the likely changes, with a reference to the related assumption (aref), as appropriate.}

\end{itemize}

\section{Unlikely Changes}    

\noindent \begin{itemize}

\item[LC\refstepcounter{lcnum}\thelcnum\label{LC_meaningfulLabel}:] \plt{Give
    the unlikely changes.  The design can assume that the changes listed will
    not occur.}

\end{itemize}

\section{Development Plan}

The development plan is categorized into multiple sections, where each section represented a significant phase in the progress of project execution. A section is given a number in the hundreds (X00) to denote a significant phase in the project. Each section is subdivided further into segments given by numbers specified in the tens (XX0) to denote smaller steps within each phase. The expected order of segment completion follows the order of increasing number count; the lowest number segment should be completed first and the highest number segment should be completed last.\\

Each segment has an overall goal that can include the coordination of multiple teammates. Upon completion of each segment, the team members relevant to the segment review and buyoff the readyness of the segment. Upon completion of buying off each segment within a section, the overall phase is considered to be bought off and completed with confidence. The relevant stakeholders must aim to buyoff each segment in a phase before the phase deadline.\\



Phase 1: Preperation (100 series)\\
Phase 1 Deadline: October 28, 2022\\

\begin{table}[H]
  \centering
  \begin{tabular}{|p{2cm}|p{10cm}|p{2cm}|}
  \hline
  \multicolumn{1}{|c|}{\textbf{100 Buyoffs}} & \multicolumn{1}{c|}{\textbf{Explanation}} & \multicolumn{1}{|c|}{\textbf{Stakeholder(s)}}
  \\ \hline
  110
  & Purchase sensor equipment, data measurement hardware, 3D print material.
  & Stephen
  \newline                                
  \\ \hline

  120                              
  & Obtain licenses for 3D CAD software use and database access
  & Stephen
  \newline                                
  \\ \hline

  130                          
  & Document material costs and licensing constraints
  & Stephen
  \newline                                
  \\ \hline

  140                                
  & Distribute materials and licensing to relevant project area Stakeholder
  & Stephen 
  \newline                            
  \\ \hline

  150                                
  & Completion of tool chassis mechanical design and modelling
  & Stephen 
  \newline                            
  \\ \hline

  160                                
  & Completion of electrical connection hardware circuit design and schematic
  & Stephen 
  \newline                            
  \\ \hline

  190                                
  & Tool chassis manufactured
  & Stephen 
  \newline                            
  \\ \hline

  \end{tabular}
\end{table}
\newpage

Phase 2: Proof of Concept (200 series)\\
Phase 2 Deadline: November 11, 2022\\
\begin{table}[H]
  \centering
  \begin{tabular}{|p{2cm}|p{10cm}|p{2cm}|}
  \hline
  \multicolumn{1}{|c|}{\textbf{200 Buyoffs}} & \multicolumn{1}{c|}{\textbf{Explanation}} & \multicolumn{1}{|c|}{\textbf{Stakeholder(s)}}
  \\ \hline
  210
  & Desktop application program developed with basic user interface
  & Stephen
  \newline                                
  
  \\ \hline
  220                              
  & Desktop application program can recieve data from data measurement device using a wired connection
  & Stephen
  \newline                                

  \\ \hline
  230                          
  & Desktop application program can interface with database to send data
  & Stephen
  \newline                                

  \\ \hline
  240                                
  & Desktop application program can edit data from data measurement device before sending it to the database
  & Stephen 
  \newline                            

    \\ \hline
  250                                
  & Visualization application can pull data and generate KPI metrics from the database
  & Stephen 
  \newline 

  \\ \hline
  260                                
  & Integration between data measurement device and desktop application
  & Stephen 
  \newline 

  \\ \hline
  270                                
  & Integration between desktop application and visualization application
  & Stephen 
  \newline 

  \\ \hline
  290                                
  & Integration between data measurement device, desktop application, and data measurement device
  & Stephen 
  \newline 
  \\ \hline


  \end{tabular}
\end{table}

Phase 3: Revision 0 Presentation (300 series)\\
Phase 3 Deadline: February 3, 2023\\

\begin{table}[H]
  \centering
  \begin{tabular}{|p{2cm}|p{10cm}|p{2cm}|}
  \hline
  \multicolumn{1}{|c|}{\textbf{300 Buyoffs}} & \multicolumn{1}{c|}{\textbf{Explanation}} & \multicolumn{1}{|c|}{\textbf{Stakeholder(s)}}
  \\ \hline
  310
  & Mechanical design and modelling completion of physical user interface components on tool chassis and connection modules
  & Stephen
  \newline                                
  \\ \hline

  320                              
  & Completion of wireless communication between data measurement device and desktop application 
  & Stephen
  \newline                                
  \\ \hline

  330                          
  & Completion of database security against tests that break utility of database
  & Stephen
  \newline                                
  \\ \hline

  390                                
  & Completion of extended KPI features for visualization application
  & Stephen 
  \newline                            
  \\ \hline

  \end{tabular}
\end{table}
\newpage

Phase 4: Final Demonstrations (400 series)\\
Phase 4 Deadline: March 17, 2023\\






\newpage

\bibliographystyle {plainnat}
\bibliography {../../refs/References}

\newpage

\noindent \plt{The following is not part of the template, just some things to consider
  when filing in the template.}

\noindent \plt{Grammar, flow and \LaTeX advice:
\begin{itemize}
\item For Mac users \texttt{*.DS\_Store} should be in \texttt{.gitignore}
\item \LaTeX{} and formatting rules
\begin{itemize}
\item Variables are italic, everything else not, includes subscripts (link to
  document)
\begin{itemize}
\item \href{https://physics.nist.gov/cuu/pdf/typefaces.pdf}{Conventions}
\item Watch out for implied multiplication
\end{itemize}
\item Use BibTeX
\item Use cross-referencing
\end{itemize}
\item Grammar and writing rules
\begin{itemize}
\item Acronyms expanded on first usage (not just in table of acronyms)
\item ``In order to'' should be ``to''
\end{itemize}
\end{itemize}}

\noindent \plt{Advice on using the template:
\begin{itemize}
\item Difference between physical and software constraints
\item Properties of a correct solution means \emph{additional} properties, not
  a restating of the requirements (may be ``not applicable'' for your problem).
  If you have a table of output constraints, then these are properties of a
  correct solution.
\item Assumptions have to be invoked somewhere
\item ``Referenced by'' implies that there is an explicit reference
\item Think of traceability matrix, list of assumption invocations and list of
  reference by fields as automatically generatable
\item If you say the format of the output (plot, table etc), then your
  requirement could be more abstract
\end{itemize}
}

\end{document}